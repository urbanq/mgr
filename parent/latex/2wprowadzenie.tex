\chapter{Wprowadzenie}
\label{cha:wprowadzenie}

JGP - Ideą tego rozwiązania jest stworzenie w miarę prostych i łatwych w stosowaniu metod kwalifikowania (już po wykonaniu) danego świadczenia do pewnej grupy ze ściśle zdefiniowanej listy celem rozliczenia usługi przed płatnikiem (NFZ).
Uwzględniając fakt, że aplikacja jest przeznaczona dla sektora medycznego staram się jasno i przejrzyście wytłumaczyć poszczególne pojęcia medyczne. Dla czytelnika, który nie jest zapoznany z nomenklaturą medyczną wprowadzam słownik pojęć medycznych, do którego może wracać w trakcie czytania pracy.
Zrozumienie tych pojęć jest warunkiem koniecznym, aby mieć jasne spojrzenie na tematy poruszane w dalszej części pracy. Podstawowe pojęcia używane w pracy są zdefiniowane w podrozdziałach poniżej.  

%---------------------------------------------------------------------------

\section{Rekord pacjenta}
\label{sec:rekordPacjenta}

Zbiór danych identyfikujących jednoznacznie pacjenta. Podstawowe dane to imię, nazwisko oraz numer PESEL. Z punktu widzenia pracy dane rekordu pacjenta takie jak adres, numer telefonu, historia leczenia są pomijane. Dla algorytmu grupera wymagane jest podanie daty urodzenia oraz płci pacjenta. Dane te można wydobyć z numeru PESEL.

Przykład:
Mateusz, Urbanik, 86060211756, 02.06.1986, mężczyzna

%---------------------------------------------------------------------------

\section{Katalog kodów rozpoznań i procedur (ICD-10, ICD9)}
\label{sec:kodyICD}

Podstawowe dane określające przebieg leczenia pacjenta to rozpoznania i procedury. Rozpoznania medyczne zostały sklasyfikowane w katalogu ICD-10. NFZ definiuje kalog ICD-10 jako 'Międzynarodowa Statystyczna Klasyfikacja Chorób i Problemów Zdrowotnych'.
Procedury medyczne zostały sklasyfikowane w katalogu ICD-9. Lista kodów w tej pracy zawiera 3717 procedur medycznych oraz 7488 rozpoznań medycznych(diagnoz).

Przykłady rozpoznań(kod - nazwa):
\begin{itemize}
\item L23.4 - ALERGICZNE KONTAKTOWE ZAPALENIE SKÓRY WYWOŁANE BARWNIKAMI
\item I21.4 - OSTRY ZAWAŁ SERCA PODWSIERDZIOWY
\item Q72.6 - PODŁUŻNE ZNIEKSZTAŁCENIE ZMNIEJSZAJĄCE KOŚCI STRZAŁKOWEJ
\end{itemize}

Przykłady procedur(kod - nazwa):
\begin{itemize}
\item 53.591 - OPERACJA PRZEPUKLINY NADBRZUSZA
\item 78.422 - ZABIEG NAPRAWCZY ZŁEGO ZROSTU LUB BRAKU ZROSTU - KOŚĆ RAMIENNA
\item 77.66 - MIEJSCOWE WYCIĘCIE ZMIANY LUB TKANKI KOŚCI - RZEPKA
\end{itemize}

%---------------------------------------------------------------------------

\section{Hospitalizacja}
\label{sec:hospitalizacja}

Hospitalizacja, częściej nazywana epizodem lub danymi epizodu. Epizodem definiujemy leczenie w szpitalu obejmujące wszystkie świadczenia udzielone od momentu przyjęcia do momentu wypisu lub zgonu niezależnie od ilości oddziałów(pobytów), w których pacjent był leczony. Pobyt oznacza leczenie w oddziale o określonej specjalności.

Przykład:
\begin{itemize}
\item data urodzenia	: 02.06.1986
\item płeć		: mężczyzna
\item data przyjęcia	: 01.05.2012
\item data wypisu	: 15.05.2012
\item tryb przyjęcia	: Przyjęcie planowe na podstawie skierowania
\item tryb wypisu	: Zakończenie procesu terapeutycznego lub diagnostycznego
\item tryb hospitalizacji : Hospitalizacja zwykła
\item pobyt		:
 \begin{itemize}
 \item oddział		: kardiologia
 \item kod świadczenia	: 0.1 - Leczenie stacjonarne - Pobyt na oddziale szpitalnym
 \item data przyjęcia	: 01.05.2012
 \item data wypisu	: 15:05.2012
 \item rozpoznanie zasadnicze	: I00 - CHOROBA REUMATYCZNA SERCA BEZ WZMIANKI O ZAJĘCIU SERCA
 \item procedura znacząca	: 37.271 - MAPOWANIE SERCA Z WYKORZYSTANIEM SYSTEMU ELEKTROANATOMICZNEGO - wykonano 17.05.2012
 \item procedura dodatkowa	: 37.49 - INNE ZABIEGI NAPRAWCZE SERCA I OSIERDZIA - wykonanno 17.05.2012
 \end{itemize}
\end{itemize}

%---------------------------------------------------------------------------

\section{Charakterystyka JGP}
\label{sec:charakterystykaJGP}

Charakterystyka JGP jest to zbiór wszystkich istotnych parametrów służących do prawidłowego wyznaczenia grupy:
\begin{itemize}
\item powiązana grupa JGP
\item rozpoznania zasadnicze, współistniejące według kodyfikacji ICD-10
\item wykonane istotne procedury diagnostyczne lub lecznicze według klasyfikacji ICD-9
\item rozpoznania i procedury wykluczające się
\item wiek pacjenta ustalany na podstawie numeru PESEL lub daty urodzenia
\item czas pobytu w szpitalu
\item tryb przyjęcia, tryb wypisu
\item płeć
\end{itemize}

%---------------------------------------------------------------------------

\subsection{Kod JGP}
\label{sec:kodJGP}

JGP, czyli Jednorodne Grupy Pacjentów jest to tłumaczenie angielskiego terminu DRG - Diagnosis Related Groups.
JGP posiada swój własny unikalny kod, kod produktu oraz nazwę. Każda z grup posiada wartości punktowe wyliczane przez płatnika, zależne od trybu hospitalizacji. W mojej pracy bazuję na liście 514 kodów JGP.

Przykład:
\begin{itemize}
\item A50 - Udar mózgu - leczenie
\item G37 - Ostre zapalenie trzustki
\item P24 - Cukrzyca
\end{itemize}

%---------------------------------------------------------------------------

\subsection{Warunki kierunkowe}
\label{sec:warunkiKierunkowe}

Dla każdej grupy zdefiniowany jest zestaw warunków logicznych, które pozwalają na przypisanie episodu do tej grupy JGP. Są to dodatkowe wymagania, które decydują o przebiegu grupowania. W algorytmie grupera w wersji 7 zdefiniowanych jest 26 warunków kierunkowych. Oznaczone są one kolejno literami alfabetu.
Przykład:
\begin{itemize}
\item D - grupa zdefiniowana rozpoznaniem i dwiema procedurami, z jednej listy, może mieć dodatkowy warunek (czas hospitalizacji)
\item R - warunek występuje w grupie zdefiniowanej także innym warunkiem kierunkowym; rozpoznanie 
z listy grupy musi występować jako rozpoznanie współistniejące i być potwierdzone odpowiednim rozpoznaniem zasadniczym z listy ogólnej; może mieć dodatkowy warunek (drugie rozpoznanie współistniejące)
\item X - grupa zdefiniowana rozpoznaniem zasadniczym i rozpoznaniem współistniejącym z listy dodatkowej oraz procedurą z listy dodatkowej; dodatkowe warunki (czas hospitalizacji, wiek)
\end{itemize}

%---------------------------------------------------------------------------

\subsection{Warunki dodatkowe}
\label{sec:warunkiDodatkowe}

W przebiegu grupowania brane są pod uwagę warunki dodatkowe takie jak: wiek, czas hospitalizacji, płeć, tryb przyjęcia, tryb wypisu, oddział. Zdefiniowane ograniczenia powodują zaliczenie lub nie zaliczenie grupy dla danego epizodu.
Przykład:
\begin{itemize}
\item Czas hospitalizacji mniejszy od 14 dni.
\item Wiek pacjenta większy niż 2 tygodnie i mniejszy niż 3 miesiące
\item Leczenie przebiegało na oddziale urologii
\end{itemize}

%---------------------------------------------------------------------------

\section{Gruper NFZ}
\label{sec:gruperNFZ}

Przez pojęcie Gruper lub 'Gruper JGP' rozumiemy aplikację umożliwiającą kwalifikację rekordu pacjenta (na podstawie danych epizodu) do właściwej grupy systemu JGP. Wynikiem grupowania jest konkretna grupa z katalogu jednorodnych grup pacjentów oraz wartość punktowa, która jest przeliczana przez płatnika na konkretną sumę pieniężną.

%---------------------------------------------------------------------------

\section{Publikacje NFZ}
\label{sec:publikacjeNFZ}

Narodowy fundusz zdrowia publikuje na swojej witrynie internetowej \underline{\texttt{www.nfz.gov.pl}} wszystkie potrzebne pliki potrzebne do stworzenia aplikacji typu gruper. Pomijając pliki z opisami zmian, prezentacje na temat procesu grupowania oraz inne mało istotne istnieją 2 zasadnicze pliki. Są to: plik parametryzujący, oraz algorytm grupera. Do dzisiaj wyszło ponad 30 wersji pliku parametryzującego oraz ponad 10 wersji pliku z algorytmem grupera.
Plik parametryzujący jest to dokument w arkuszu MS-Excel zawierający wszystkie potrzebne dla grupera dane w niezestandaryzowanym zapisie. Natomiast algorytm grupera jest to dokument w formacie MS-Word opisujący algorytm grupera według wymogów Narodowoego Funuszu Zdrowia. Jest to 34 stronnicowy opis algorytmu napisany przez urzędników Państwowych, którzy nie posiadają wiedzy z zakresu standardów takich jak UML. Zrozumienie tak sformułowanego opisu skomplikowanego algorytmu grupowania to prawdziwe wyzwanie dla studenta informatyki nawet z 3-letnim doświadczeniem w branży medycznej. Zbiór danych na których postanowiłem pracować to jedna z najnowszych wersji pliku parametryzującego (wersja 25). Algorytm gurpera postanowiłem zaimplementować w wersji 7. Różnice pomiędzy wersjami plików parametryzująch oraz plików z algorytmem są subtelne i nie mają najmniejszego wpływu na wynik pracy.

%---------------------------------------------------------------------------

\section{Po co system JGP?}
\label{sec:poCoJGP}

System Jednorodnych Grup Pacjentów został wprowadzony, aby usystematyzować w Polsce sposób rozliczania hospitalizacji pacjenta. Praktyczna konstrukcja tego systemu wynika z obserwacji, że pewne grupy pacjentów, często znacznie różniące się wymagają dość podobnego postępowania. Natomiast z drugiej strony, ta sama choroba u pacjentów rożniących się wiekiem i współistniejącymi problemami wymaga często innego postępowania.
Zwróćmy uwagę, że ze zbioru 3717 procedur oraz 7488 rozpoznań, które w danych episodu mogą występować w niezliczonej ilości kombinacji w wyniku procesu grupowania otrzymujemy zawsze jedną grupę spośród 514 możliwych oraz dokładnie jedną wartość punktową, która jednoznacznie definiuje koszty leczenia.
Wartości punktowe dla poszczególnych grup, czyli wycena grup została dokonana przez ekspertów Narodowego Funduszu Zdrowia na podstawie danych sprawozdawczych z monitorowanych szpitali w okresie 2002-2007. Obecnie wycena opiera się na wszystkich danych sprawozdawczych przekazywanych przez wszystkie szpitale oddziałom NFZ i dostosowywana jest do rzeczywistych kosztów, z częstotliwością co około 6 miesięcy.


\vspace*{1cm}
\includegraphics{images/standarization}

%---------------------------------------------------------------------------

\section{Historia systemu Jednorodnych Grup Pacjentów}
\label{sec:historiaJGP}

Początki systemu JGP sięgają lat 60-tych XX wieku systemu opieki zdrowotnej w USA. Wtedy zaobserwowano, iż pacjenci z tym samym schorzeniem, którym wykonywano te same badania i procedury oraz których czas hospitalizacji był zbliżony, generowali te same koszty. Na przełomie lat 60. i 70. prof. Robert Fetter z Uniwersytetu Yale po raz pierwszy przedstawił założenia systemu Jednorodnych Grup Pacjentów (Diagnosis Related Groups – DRG). Pacjenci podobni medycznie i kosztowo zostali przyporządkowani do 333 grup, w 54 głównych kategoriach diagnostycznych. Prof. Robert Fetter oparł się na analizie danych 1 mln 700 tys. pacjentów hospitalizowanych w szpitalach stanu New Jersey. Ostatecznie w Stanach Zjednoczonych system DRG wprowadzono w życie decyzją Kongresu w roku 1982.

Zaproponowany system zaczął się upowszechniać w Europie pod koniec lat 90. XX wieku. W Polsce pierwsze próby implantacji modelu DRG miały miejsce jeszcze w czasach Kas Chorych:
\begin{enumerate}
\item Grupy JGP w ginekologii i położnictwie w Łódzkiej Kasie Chorych w 1999 roku, opracowane na podstawie danych o kosztach leczenia pacjentek z 10 szpitali z województwa warmińsko-mazurskiego
\item Wdrożenie systemu JGP do rozliczeń świadczeń szpitalnych w Dolnośląskiej i Podkarpackiej Kasie Chorych w latach 1999 - 2003
\item Projekt adaptacji austriackiego systemu LKF zrealizowany w ramach projektu Banku Światowego w latach 2000 – 2002
\item Doświadczenia zebrane w trakcie trwania projektu VITAPOL komponent-3: „Przegląd polskiego systemu ustalania kosztów w opiece zdrowotnej” – umowa twinningowa realizowana przez brytyjskich ekspertów.
\end{enumerate} 
To właśnie brytyjski system HRG stał się podstawą opracowania polskiego modelu Jednorodnych Grup Pacjentów JGP. System Jednorodnych Grup Pacjentów w leczeniu szpitalnym został wprowadzony w Polsce 1 lipca 2008, a jego głównym twórcą jest dr Jacek Grabowski, ekspert systemów opieki zdrowotnej, z wykształcenia lekarz psychiatra.

W chwili obecnej system JGP jest zalecany przez Komisję Europejską na terenie całej Unii Europejskiej. Rozpoczęto prace nad projektem Euro-DRG 7, który docelowo stanowiłby jednolity system JGP dla całej Unii Europejskiej.

