\documentclass[pdflatex,11pt]{aghdpl}
% \documentclass{aghdpl}               % przy kompilacji programem latex
% \documentclass[pdflatex,en]{aghdpl}  % praca w języku angielskim
\usepackage[polish]{babel}
\usepackage[utf8]{inputenc}

% dodatkowe pakiety
\usepackage{enumerate}
\usepackage{listings}
\usepackage{color}
\usepackage{comment}

\lstloadlanguages{TeX, Java, SQL}

\lstset{
  literate={ą}{{\k{a}}}1
           {ć}{{\'c}}1
           {ę}{{\k{e}}}1
           {ó}{{\'o}}1
           {ń}{{\'n}}1
           {ł}{{\l{}}}1
           {ś}{{\'s}}1
           {ź}{{\'z}}1
           {ż}{{\.z}}1
           {Ą}{{\k{A}}}1
           {Ć}{{\'C}}1
           {Ę}{{\k{E}}}1
           {Ó}{{\'O}}1
           {Ń}{{\'N}}1
           {Ł}{{\L{}}}1
           {Ś}{{\'S}}1
           {Ź}{{\'Z}}1
           {Ż}{{\.Z}}1
}

\definecolor{dkgreen}{rgb}{0,0.6,0}
\definecolor{gray}{rgb}{0.5,0.5,0.5}
\definecolor{mauve}{rgb}{0.58,0,0.82}
 
\lstset{ % Java
  language=Java,                  % the language of the code
  basicstyle=\footnotesize,       % the size of the fonts that are used for the code
  numbers=left,                   % where to put the line-numbers
  numberstyle=\tiny\color{gray},  % the style that is used for the line-numbers
  stepnumber=1,                   % the step between two line-numbers. If it's 1, each line 
                                  % will be numbered
  numbersep=5pt,                  % how far the line-numbers are from the code
  backgroundcolor=\color{white},  % choose the background color. You must add \usepackage{color}
  showspaces=false,               % show spaces adding particular underscores
  showstringspaces=false,         % underline spaces within strings
  showtabs=false,                 % show tabs within strings adding particular underscores
  frame=single,                   % adds a frame around the code
  rulecolor=\color{black},        % if not set, the frame-color may be changed on line-breaks within not-black text (e.g. commens (green here))
  tabsize=2,                      % sets default tabsize to 2 spaces
  captionpos=b,                   % sets the caption-position to bottom
  breaklines=true,                % sets automatic line breaking
  breakatwhitespace=false,        % sets if automatic breaks should only happen at whitespace
  title=\lstname,                 % show the filename of files included with \lstinputlisting;
                                  % also try caption instead of title
  keywordstyle=\color{blue},      % keyword style
  commentstyle=\color{dkgreen},   % comment style
  stringstyle=\color{mauve},      % string literal style
  escapeinside={\%*}{*)},         % if you want to add LaTeX within your code
  morekeywords={*,...},           % if you want to add more keywords to the set
}


\lstset{ % SQL
  language=SQL,                   % the language of the code
  basicstyle=\footnotesize,       % the size of the fonts that are used for the code
  numbers=none,                   % where to put the line-numbers
  numberstyle=\tiny\color{gray},  % the style that is used for the line-numbers
  stepnumber=1,                   % the step between two line-numbers. If it's 1, each line 
                                  % will be numbered
  numbersep=5pt,                  % how far the line-numbers are from the code
  backgroundcolor=\color{white},  % choose the background color. You must add \usepackage{color}
  showspaces=false,               % show spaces adding particular underscores
  showstringspaces=false,         % underline spaces within strings
  showtabs=false,                 % show tabs within strings adding particular underscores
  frame=single,                   % adds a frame around the code
  rulecolor=\color{black},        % if not set, the frame-color may be changed on line-breaks within not-black text (e.g. commens (green here))
  tabsize=2,                      % sets default tabsize to 2 spaces
  captionpos=b,                   % sets the caption-position to bottom
  breaklines=true,                % sets automatic line breaking
  breakatwhitespace=false,        % sets if automatic breaks should only happen at whitespace
  title=\lstname,                 % show the filename of files included with \lstinputlisting;
                                  % also try caption instead of title
  keywordstyle=\color{blue},      % keyword style
  commentstyle=\color{dkgreen},   % comment style
  stringstyle=\color{mauve},      % string literal style
  escapeinside={\%*}{*)},         % if you want to add LaTeX within your code
  morekeywords={*,...},           % if you want to add more keywords to the set
  otherkeywords={CSVREAD},
  morecomment=[l]--,
  extendedchars=\true,
}


%---------------------------------------------------------------------------

\author{Mateusz Urbanik}
\shortauthor{M. Urbanik}

\titlePL{Opracowanie metod optymalizacji kosztów leczenia pacjenta przy użyciu technologii informatycznych}
\titleEN{Developing methods to optimize patient treatment costs, using information technology}

\shorttitlePL{Optymalizacja kosztów leczenia pacjenta} % skrócona wersja tytułu jeśli jest bardzo długi
\shorttitleEN{Optimization of patient treatment costs}

\thesistypePL{Praca magisterska}
\thesistypeEN{Master of Science Thesis}

\supervisorPL{dr inż. Rafał Mrówka}
\supervisorEN{Rafał Mrówka D.}

\date{2012}

\departmentPL{Katedra Automatyki}
\departmentEN{Department of Automatics}

\facultyPL{Wydział Elektrotechniki, Automatyki, Informatyki i Elektroniki}
\facultyEN{Faculty of Electrical Engineering, Automatics, Computer Science and Electronics}

\acknowledgements{Serdecznie dziękuję promotorowi za wsparcie podczas pisania tej pracy oraz cenne wskazówki.}



\setlength{\cftsecnumwidth}{10mm}

%---------------------------------------------------------------------------

\begin{document}

\titlepages

\tableofcontents
\clearpage

\chapter{Wstęp}
\label{cha:wstep}

Rozliczanie kosztów leczenia pacjenta jest problemem skomplikowanym. Z jednej strony dążymy do rozwiązania, w~którym kluczowym czynnikiem jest zdrowie pacjenta. Aspekt medyczny brany jest pod uwagę na pierwszym miejscu. Jednak z drugiej strony równie ważnym jest aspekt finansowy. Nieprawidłowo działający system rozliczeń ma bezpośredni wpływ na aspekt medyczny, co przekłada się na niską jakość świadczonych usług. Należy więc oczekiwać, że przyjęta przez płatnika metoda finansowania świadczeń opieki zdrowotnej powinna dążyć do zapewnienia obiektywnego pomiaru kosztów leczenia w~jednostkach szpitalnych. Jednocześnie powinna motywować świadczeniodawców do zwiększania dostępności do odpowiedniej jakości świadczeń, jak również ograniczać nieuzasadniony wzrost kosztów opieki zdrowotnej.

Dotychczasowe doświadczenia światowe wskazują, że najlepszą metodą finansowania świadczeń medycznych jest system Jednorodnych Grup Pacjentów. Łączy on zarówno aspekt medyczny jak i finansowy. Istotnym założeniem tego systemu jest jego otwarty charakter. Corocznie każdy z krajów, w~których obowiązuje system JGP przyjmuje nową wersję systemu, który jest ciągle doskonalony i rozwijany.

%---------------------------------------------------------------------------

\section{Cele pracy}
\label{sec:celePracy}

Celem poniższej pracy jest stworzenie aplikacji typu gruper według wytycznych Narodowego Funduszu Zdrowia. Aplikacja będzie systemem ekspertowym opartym na bazie wiedzy dostarczonej przez NFZ. Algorytm grupowania będzie własną implementacją systemu regułowego w~języku JAVA, który będzie odpowiadał algorytmowi zapropowanemu przez NFZ. Program będzie jedynym rozwiązaniem typu ,,open source'' na rynku. Poruszony zostanie również problem optymalizacji procesu grupowania, czyli maksymalizacji kosztów leczenia pacjenta. Należy podkreślić, że każda placówka medyczna chce dostać jak największy zwrot poniesionych środków od płatnika. Na końcu pracy przedstawione zostaną realne przypadki użycia systemu dla konkretnych przypadków leczenia. Napisane zostaną testy integracyjne, sprawdzające poprawność działania algorytmu grupera dla wszystkich zdefiniowanych reguł. 
Podsumowując stawiane cele:
\begin{enumerate}
\item Implementacja algorytmu ,,Gruper JGP''.
\item Stworzenie własnego systemu ekspertowego dla rozwiązania.
\item Optymalizacja kosztów leczenia.
\item Rozwiązanie typu ,,open source''.
\item Testy integracyjne, realne przypadki użycia.
\end{enumerate}


%---------------------------------------------------------------------------

\section{Teza pracy}
\label{sec:tezaPracy}
//TODO

%---------------------------------------------------------------------------

\section{Zawartość pracy}
\label{sec:zawartoscPracy}

W rodziale~\ref{cha:wprowadzenie} przedstawiono wszystkie podstawowe informacje oraz zdefiniowano słownik podstawowych pojęć. Wyjaśniona została istota aplikacji grupera oraz powszechnego na całym świecie systemu Jednorodnych Grup Pacjentów. Przedstawiona została również bardzo skrócona wersja historii powstania systemu JGP.

W rodziale~\ref{cha:rozwiazanie} został szczegółowo opisany system, który został stworzony w~ramach tej pracy. W~tym rozdziale znajduje się opis architektury systemu, modelu danych, następnie wyjaśniony został szczegółowo algorytm grupowania. Opis systemu ekspertowego własnej implementacji odpowiadającego algorytmowi grupowania znajduje się w~podrozdziale~\ref{sec:systemEkspertowy}.

Rozdział~\ref{cha:prezentacja} zawiera prezentację działania systemu na konkretnym przykładzie. Przedstawiony został przypadek użycia aplikacji, który zawiera rozliczenie kosztów hospitalizacji oraz przeprowadzenie optymalizacji otrzymanych wyników.

Podsumowanie pracy znajduje się w~rozdziale ~\ref{cha:podsumowanie} - są tam informacje o możliwościach rozwoju aplikacji; porównanie z istniejącymi na rynku rozwiązaniami oraz zestawienie wszystkich użytych narzędzi i bibliotek.


\chapter{Wprowadzenie}
\label{cha:wprowadzenie}

JGP - Ideą tego rozwiązania jest stworzenie w miarę prostych i łatwych w stosowaniu metod kwalifikowania (już po wykonaniu) danego świadczenia do pewnej grupy ze ściśle zdefiniowanej listy celem rozliczenia usługi przed płatnikiem (NFZ).
Uwzględniając fakt, że aplikacja jest przeznaczona dla sektora medycznego staram się jasno i przejrzyście wytłumaczyć poszczególne pojęcia medyczne. Dla czytelnika, który nie jest zapoznany z nomenklaturą medyczną wprowadzam słownik pojęć medycznych, do którego może wracać w trakcie czytania pracy.
Zrozumienie tych pojęć jest warunkiem koniecznym, aby mieć jasne spojrzenie na tematy poruszane w dalszej części pracy. Podstawowe pojęcia używane w pracy są zdefiniowane w podrozdziałach poniżej.  

%---------------------------------------------------------------------------

\section{Rekord pacjenta}
\label{sec:rekordPacjenta}

Zbiór danych identyfikujących jednoznacznie pacjenta. Podstawowe dane to imię, nazwisko oraz numer PESEL. Z punktu widzenia pracy dane rekordu pacjenta takie jak adres, numer telefonu, historia leczenia są pomijane. Dla algorytmu grupera wymagane jest podanie daty urodzenia oraz płci pacjenta. Dane te można wydobyć z numeru PESEL.

Przykład:
Mateusz, Urbanik, 86060211756, 02.06.1986, mężczyzna

%---------------------------------------------------------------------------

\section{Katalog kodów rozpoznań i procedur (ICD-10, ICD9)}
\label{sec:kodyICD}

Podstawowe dane określające przebieg leczenia pacjenta to rozpoznania i procedury. Rozpoznania medyczne zostały sklasyfikowane w katalogu ICD-10. NFZ definiuje kalog ICD-10 jako 'Międzynarodowa Statystyczna Klasyfikacja Chorób i Problemów Zdrowotnych'.
Procedury medyczne zostały sklasyfikowane w katalogu ICD-9. Lista kodów w tej pracy zawiera 3717 procedur medycznych oraz 7488 rozpoznań medycznych(diagnoz).

Przykłady rozpoznań(kod - nazwa):
\begin{itemize}
\item L23.4 - ALERGICZNE KONTAKTOWE ZAPALENIE SKÓRY WYWOŁANE BARWNIKAMI
\item I21.4 - OSTRY ZAWAŁ SERCA PODWSIERDZIOWY
\item Q72.6 - PODŁUŻNE ZNIEKSZTAŁCENIE ZMNIEJSZAJĄCE KOŚCI STRZAŁKOWEJ
\end{itemize}

Przykłady procedur(kod - nazwa):
\begin{itemize}
\item 53.591 - OPERACJA PRZEPUKLINY NADBRZUSZA
\item 78.422 - ZABIEG NAPRAWCZY ZŁEGO ZROSTU LUB BRAKU ZROSTU - KOŚĆ RAMIENNA
\item 77.66 - MIEJSCOWE WYCIĘCIE ZMIANY LUB TKANKI KOŚCI - RZEPKA
\end{itemize}

%---------------------------------------------------------------------------

\section{Hospitalizacja}
\label{sec:hospitalizacja}

Hospitalizacja, częściej nazywana epizodem lub danymi epizodu. Epizodem definiujemy leczenie w szpitalu obejmujące wszystkie świadczenia udzielone od momentu przyjęcia do momentu wypisu lub zgonu niezależnie od ilości oddziałów(pobytów), w których pacjent był leczony. Pobyt oznacza leczenie w oddziale o określonej specjalności.

Przykład:
\begin{itemize}
\item data urodzenia	: 02.06.1986
\item płeć		: mężczyzna
\item data przyjęcia	: 01.05.2012
\item data wypisu	: 15.05.2012
\item tryb przyjęcia	: Przyjęcie planowe na podstawie skierowania
\item tryb wypisu	: Zakończenie procesu terapeutycznego lub diagnostycznego
\item tryb hospitalizacji : Hospitalizacja zwykła
\item pobyt		:
 \begin{itemize}
 \item oddział		: kardiologia
 \item kod świadczenia	: 0.1 - Leczenie stacjonarne - Pobyt na oddziale szpitalnym
 \item data przyjęcia	: 01.05.2012
 \item data wypisu	: 15:05.2012
 \item rozpoznanie zasadnicze	: I00 - CHOROBA REUMATYCZNA SERCA BEZ WZMIANKI O ZAJĘCIU SERCA
 \item procedura znacząca	: 37.271 - MAPOWANIE SERCA Z WYKORZYSTANIEM SYSTEMU ELEKTROANATOMICZNEGO - wykonano 17.05.2012
 \item procedura dodatkowa	: 37.49 - INNE ZABIEGI NAPRAWCZE SERCA I OSIERDZIA - wykonanno 17.05.2012
 \end{itemize}
\end{itemize}

%---------------------------------------------------------------------------

\section{Charakterystyka JGP}
\label{sec:charakterystykaJGP}

Charakterystyka JGP jest to zbiór wszystkich istotnych parametrów służących do prawidłowego wyznaczenia grupy:
\begin{itemize}
\item powiązana grupa JGP
\item rozpoznania zasadnicze, współistniejące według kodyfikacji ICD-10
\item wykonane istotne procedury diagnostyczne lub lecznicze według klasyfikacji ICD-9
\item rozpoznania i procedury wykluczające się
\item wiek pacjenta ustalany na podstawie numeru PESEL lub daty urodzenia
\item czas pobytu w szpitalu
\item tryb przyjęcia, tryb wypisu
\item płeć
\end{itemize}

%---------------------------------------------------------------------------

\subsection{Kod JGP}
\label{sec:kodJGP}

JGP, czyli Jednorodne Grupy Pacjentów jest to tłumaczenie angielskiego terminu DRG - Diagnosis Related Groups.
JGP posiada swój własny unikalny kod, kod produktu oraz nazwę. Każda z grup posiada wartości punktowe wyliczane przez płatnika, zależne od trybu hospitalizacji. W mojej pracy bazuję na liście 514 kodów JGP.

Przykład:
\begin{itemize}
\item A50 - Udar mózgu - leczenie
\item G37 - Ostre zapalenie trzustki
\item P24 - Cukrzyca
\end{itemize}

%---------------------------------------------------------------------------

\subsection{Warunki kierunkowe}
\label{sec:warunkiKierunkowe}

Dla każdej grupy zdefiniowany jest zestaw warunków logicznych, które pozwalają na przypisanie episodu do tej grupy JGP. Są to dodatkowe wymagania, które decydują o przebiegu grupowania. W algorytmie grupera w wersji 7 zdefiniowanych jest 26 warunków kierunkowych. Oznaczone są one kolejno literami alfabetu.
Przykład:
\begin{itemize}
\item D - grupa zdefiniowana rozpoznaniem i dwiema procedurami, z jednej listy, może mieć dodatkowy warunek (czas hospitalizacji)
\item R - warunek występuje w grupie zdefiniowanej także innym warunkiem kierunkowym; rozpoznanie 
z listy grupy musi występować jako rozpoznanie współistniejące i być potwierdzone odpowiednim rozpoznaniem zasadniczym z listy ogólnej; może mieć dodatkowy warunek (drugie rozpoznanie współistniejące)
\item X - grupa zdefiniowana rozpoznaniem zasadniczym i rozpoznaniem współistniejącym z listy dodatkowej oraz procedurą z listy dodatkowej; dodatkowe warunki (czas hospitalizacji, wiek)
\end{itemize}

%---------------------------------------------------------------------------

\subsection{Warunki dodatkowe}
\label{sec:warunkiDodatkowe}

W przebiegu grupowania brane są pod uwagę warunki dodatkowe takie jak: wiek, czas hospitalizacji, płeć, tryb przyjęcia, tryb wypisu, oddział. Zdefiniowane ograniczenia powodują zaliczenie lub nie zaliczenie grupy dla danego epizodu.
Przykład:
\begin{itemize}
\item Czas hospitalizacji mniejszy od 14 dni.
\item Wiek pacjenta większy niż 2 tygodnie i mniejszy niż 3 miesiące
\item Leczenie przebiegało na oddziale urologii
\end{itemize}

%---------------------------------------------------------------------------

\section{Gruper NFZ}
\label{sec:gruperNFZ}

Przez pojęcie Gruper lub 'Gruper JGP' rozumiemy aplikację umożliwiającą kwalifikację rekordu pacjenta (na podstawie danych epizodu) do właściwej grupy systemu JGP. Wynikiem grupowania jest konkretna grupa z katalogu jednorodnych grup pacjentów oraz wartość punktowa, która jest przeliczana przez płatnika na konkretną sumę pieniężną.

%---------------------------------------------------------------------------

\section{Publikacje NFZ}
\label{sec:publikacjeNFZ}

Narodowy fundusz zdrowia publikuje na swojej witrynie internetowej \underline{\texttt{www.nfz.gov.pl}} wszystkie potrzebne pliki potrzebne do stworzenia aplikacji typu gruper. Pomijając pliki z opisami zmian, prezentacje na temat procesu grupowania oraz inne mało istotne istnieją 2 zasadnicze pliki. Są to: plik parametryzujący, oraz algorytm grupera. Do dzisiaj wyszło ponad 30 wersji pliku parametryzującego oraz ponad 10 wersji pliku z algorytmem grupera.
Plik parametryzujący jest to dokument w arkuszu MS-Excel zawierający wszystkie potrzebne dla grupera dane w niezestandaryzowanym zapisie. Natomiast algorytm grupera jest to dokument w formacie MS-Word opisujący algorytm grupera według wymogów Narodowoego Funuszu Zdrowia. Jest to 34 stronnicowy opis algorytmu napisany przez urzędników Państwowych, którzy nie posiadają wiedzy z zakresu standardów takich jak UML. Zrozumienie tak sformułowanego opisu skomplikowanego algorytmu grupowania to prawdziwe wyzwanie dla studenta informatyki nawet z 3-letnim doświadczeniem w branży medycznej. Zbiór danych na których postanowiłem pracować to jedna z najnowszych wersji pliku parametryzującego (wersja 25). Algorytm gurpera postanowiłem zaimplementować w wersji 7. Różnice pomiędzy wersjami plików parametryzująch oraz plików z algorytmem są subtelne i nie mają najmniejszego wpływu na wynik pracy.

%---------------------------------------------------------------------------

\section{Po co system JGP?}
\label{sec:poCoJGP}

System Jednorodnych Grup Pacjentów został wprowadzony, aby usystematyzować w Polsce sposób rozliczania hospitalizacji pacjenta. Praktyczna konstrukcja tego systemu wynika z obserwacji, że pewne grupy pacjentów, często znacznie różniące się wymagają dość podobnego postępowania. Natomiast z drugiej strony, ta sama choroba u pacjentów rożniących się wiekiem i współistniejącymi problemami wymaga często innego postępowania.
Zwróćmy uwagę, że ze zbioru 3717 procedur oraz 7488 rozpoznań, które w danych episodu mogą występować w niezliczonej ilości kombinacji w wyniku procesu grupowania otrzymujemy zawsze jedną grupę spośród 514 możliwych oraz dokładnie jedną wartość punktową, która jednoznacznie definiuje koszty leczenia.
Wartości punktowe dla poszczególnych grup, czyli wycena grup została dokonana przez ekspertów Narodowego Funduszu Zdrowia na podstawie danych sprawozdawczych z monitorowanych szpitali w okresie 2002-2007. Obecnie wycena opiera się na wszystkich danych sprawozdawczych przekazywanych przez wszystkie szpitale oddziałom NFZ i dostosowywana jest do rzeczywistych kosztów, z częstotliwością co około 6 miesięcy.


\vspace*{1cm}
\includegraphics{images/standarization}

%---------------------------------------------------------------------------

\section{Historia systemu Jednorodnych Grup Pacjentów}
\label{sec:historiaJGP}

Początki systemu JGP sięgają lat 60-tych XX wieku systemu opieki zdrowotnej w USA. Wtedy zaobserwowano, iż pacjenci z tym samym schorzeniem, którym wykonywano te same badania i procedury oraz których czas hospitalizacji był zbliżony, generowali te same koszty. Na przełomie lat 60. i 70. prof. Robert Fetter z Uniwersytetu Yale po raz pierwszy przedstawił założenia systemu Jednorodnych Grup Pacjentów (Diagnosis Related Groups – DRG). Pacjenci podobni medycznie i kosztowo zostali przyporządkowani do 333 grup, w 54 głównych kategoriach diagnostycznych. Prof. Robert Fetter oparł się na analizie danych 1 mln 700 tys. pacjentów hospitalizowanych w szpitalach stanu New Jersey. Ostatecznie w Stanach Zjednoczonych system DRG wprowadzono w życie decyzją Kongresu w roku 1982.

Zaproponowany system zaczął się upowszechniać w Europie pod koniec lat 90. XX wieku. W Polsce pierwsze próby implantacji modelu DRG miały miejsce jeszcze w czasach Kas Chorych:
\begin{enumerate}
\item Grupy JGP w ginekologii i położnictwie w Łódzkiej Kasie Chorych w 1999 roku, opracowane na podstawie danych o kosztach leczenia pacjentek z 10 szpitali z województwa warmińsko-mazurskiego
\item Wdrożenie systemu JGP do rozliczeń świadczeń szpitalnych w Dolnośląskiej i Podkarpackiej Kasie Chorych w latach 1999 - 2003
\item Projekt adaptacji austriackiego systemu LKF zrealizowany w ramach projektu Banku Światowego w latach 2000 – 2002
\item Doświadczenia zebrane w trakcie trwania projektu VITAPOL komponent-3: „Przegląd polskiego systemu ustalania kosztów w opiece zdrowotnej” – umowa twinningowa realizowana przez brytyjskich ekspertów.
\end{enumerate} 
To właśnie brytyjski system HRG stał się podstawą opracowania polskiego modelu Jednorodnych Grup Pacjentów JGP. System Jednorodnych Grup Pacjentów w leczeniu szpitalnym został wprowadzony w Polsce 1 lipca 2008, a jego głównym twórcą jest dr Jacek Grabowski, ekspert systemów opieki zdrowotnej, z wykształcenia lekarz psychiatra.

W chwili obecnej system JGP jest zalecany przez Komisję Europejską na terenie całej Unii Europejskiej. Rozpoczęto prace nad projektem Euro-DRG 7, który docelowo stanowiłby jednolity system JGP dla całej Unii Europejskiej.


\chapter{Opis stworzonego rozwiązania}
\label{cha:rozwiazanie}

Niniejszy rozdział zawiera opis systemu, który został stworzony na potrzeby tej pracy. W rozdziale~\ref{sec:zalozeniaProjektowe} zostały krótko przedstawione najważniejsze założenia, na których opiera się realizacja projektu. Następnie rozdział~\ref{sec:architekturaSystemu} opisuje architekturę stworzonego rozwiązania. Rozdział~\ref{sec:modelDanych} zawiera szczegółowy opis w jaki sposób został określony model danych. Sposób działania algorytmu grupowania został zaprezentowany w rozdziale~\ref{sec:gruperNFZ}. Spojrzenie na system z punktu widzenia systemu ekspertowego został zawarty w sekcji~\ref{sec:systemEkspertowy}. Rozdział~\ref{sec:optymalizacjaJGP} przedstawia realizację maszyny wyjąśniającej, która jest narzędziem umożliwiającym przeprowadzenie optymalizacji JGP w systemie.

%---------------------------------------------------------------------------
%---------------------------------------------------------------------------
\section{Założenia projektowe}
\label{sec:zalozeniaProjektowe}

Na kształt aplikacji bezpośredni wpływ miały założenia projektowe postawione na początku pracy. Od nich zależały decyzje podejmowane na kolejnych etapach tworzenia systemu. Lista założeń projektowych przedstawia się następująco:

\begin{enumerate}
 \item Swtorzyć aplikację typu ,,Gruper'' spełniającą wszystkie wymagania stawiane przez NFZ.
 \item Określić model danych, ustalić bazę wiedzy.
 \item System zostanie napisany w języku JAVA, będzie to aplikacja desktopowa z dostępem do bazy wiedzy, zapisanej w jednym pliku.
 \item Zastosować architekturę warstwową.
 \item Użyć najnowszych i najlepszych narzędzi do tworzenia przyjaznego dla użytkownika GUI.
 \item Projekt architektury oprzeć na metodologii Spring.
 \item W trakcie fazy implementacji zwrócić szczególną uwagę na to, aby system spełniał założenia systemu ekspertowego. Implementacja systemu będzie zalążkiem do stworzenia bardziej rozbudowanego systemu ekspertowego dla tego rozwiązania.
 \item Używać tylko narzędzi dostępnych na licencji wolnego oprogramowania.
 \item Wykonać testy dla wszystkich głównych ścieżek poszukiwań rozwiązania.
\end{enumerate}

Stworzenie aplikacji typu ,,Gruper'' oznacza, że musi zostać zaimplementowany algorytm ,,Grupera'' zgodnie z wytycznymi narzuconymi przez NFZ. Precyzyjne określenie modelu danych jest podstawowym założeniem, bez którego niemożliwa jest dalsza praca nad projektem. Zastosowanie architektury warstwowej dla systemu zwiększa elastyczność rozwiązania i pozwala na łatwe modyfikowanie i skalowalność systemu.

Implementacja całego systemu tylko w języku JAVA umożliwia rozwijanie projektu przez osoby nie znające składni oraz zasad jakimi kierują się inne języki dedykowane dla systemów ekspertowych takie jak Prolog czy Drools. Wszystkie założenia systemu ekspertowego zostają spełnione, a implementacja będzie w 100\% autorska. Fakty i reguły będą obiektami JAVY. Unikalność, innowacyjność i łatwość wdrożenia się w projekt dla tego rozwiązania mają być jego największymi atutami.



%---------------------------------------------------------------------------
%---------------------------------------------------------------------------
\section{Architektura systemu}
\label{sec:architekturaSystemu}

Wybór architektury systemu to decyzja podejmowana w początkowej fazie tworzenia systemu. Często okazuję się ona najważniejszą decyzją, która ma ogromny wpływ na rozwój oprogramowania\cite{sienkiewicz_architektura}. Kierując się wyborem architektury postawiono na jej elastyczność oraz modularność. Obrazując system jako maszynę złożoną z wielu rożnorodnych elementów dążono do stworzenia takiej, którą można w łatwy sposób ulepszać poprzez dokładanie nowych klocków oraz naprawiać poprzez podmianę starych części nowymi i lepszymi.

%---------------------------------------------------------------------------
\subsection{Model warstwowy}
\label{sec:model3warstwowy}
Architektura systemu odzwierciedla jego logiczny podział. Standardem w praktycznie każdej aplikacji jest zastosowanie modelu architektury warstwowej\cite{sienkiewicz_architektura}. Model wprowadza podział aplikacji na trzy warstwy:
\begin{enumerate}
 \item \textbf{Warstwę danych} odpowiedzialną za poprawne przechowywanie, zapis oraz odczyt danych.
 \item \textbf{Warstwę logiki biznesowej}, której zadaniem jest koordynacja pracy aplikacji, przetwarzanie żądań użytkownika, aplikacja reguł logicznych, dokonywanie obliczeń.
 \item \textbf{Warstwa prezentacji} wykonująca tłumaczenie żądań i wyników działania aplikacji do i z języka zrozumiałego dla użytkownika.
\end{enumerate}

Wybrano ten model do zastosowania przy tworzeniu aplikacji ze względu na jego logiczną i zrozumiałą dla każdego informatyka modularność. Każda z warstw może zostać zaimplementowana w innym języku programowania, a komunikują się one poprzez interfejsy. Dużym plusem tego rozwiązania jest łatwa możliwość zmiany pewnej części systemu bez zmian w innych częściach systemu. Dla przykładu, jeśli zaszłaby potrzeba stworzenia webowego interfejsu użytkownika to wystraczy tylko zaimplementować warstwę prezentacji i podpiąć ją do warstwy logiki biznesowej.

Architekturę trójwarstwową przedstawia rysunek \ref{img:rysunek_3layer}.
\begin{figure}[!ht]
\centering	
\includegraphics[scale=0.5]{images/3layer-architect}
\caption[Rysunek przedstawiający model architektury trójwarstwowej]{Architektura trójwarstwowa. Autor: Mateusz Urbanik}
\label{img:rysunek_3layer}
\end{figure}

%---------------------------------------------------------------------------
\subsection{Środowisko Spring}
\label{sec:modelArchitekturySpring}
Bardzo popularnym i nowoczesnym środowiskiem, które wprowadza zestaw narzędzi, wzorców oraz bibliotek potrzebnych do stworzenia nowoczesnych aplikacji serwerowych(i nie tylko) jest Spring. Rozwiązanie to jest niesłusznie uważane za skomplikowane, bardzo złożone i niemożliwe do opanowania. W rzeczywistości Spring stanowi zaawansowany przykład architektury szkieletowej umożliwiającej szybkie rozwijanie dowolnie złożonych aplikacji, niekoniecznie webowych. Konstrukcja szkieletu architektury Spring powoduje, że występuje w nim bardzo niewielki stopień zależności od interfejsów Spring API. Architektura Spring obejmuje wszystkie warstwy aplikacji i podsuwa rozwiązania, które mogą być stosowane zarówno w warstwie prezentacji, jak i w warstwie integracji oraz warstwie danych. Szkielet aplikacyjny Spring jest rozwijany na licencji ,,open source'' od roku 2003 i od tego czasu zdobył sobie dużą popularność\cite{bruce_spring}. Rysunek \ref{img:rysunek_spring} ilustruje schemat architektury Spring.

\vspace*{0.1cm}
\begin{figure}[!ht]
\centering
\includegraphics[scale=0.31]{images/spring-modules}
\caption[Rysunek przedstawiający model architektury Spring]{Schemat architektury Spring. Źródło: Spring Reference Guide\cite{spring_reference}}
\label{img:rysunek_spring}
\end{figure}

Spring składa się z kilku modułów, z których każdy może być niezależnie używany w aplikacji. Oznacza to, że aplikacja może wykorzystywać wszystkie możliwości środowiska Spring, ale może także korzystać z fragmentu architektury, np. z modułu ułatwiającego dostęp do danych. Najważniejszym modułem jest Spring-Core. Jest to moduł oferujący zaawansowane opcje konfiguracji komponentów JavaBean oraz klas POJO(ang. Plain Old Java Object) i wykorzystujący technikę wstrzykiwania zależności.
W aplikacji postanowiono wykorzystać kilka bazowych elementów architektury. Oto lista modułów Spring wykorzystywanych w projekcie:
\begin{itemize}
 \item JDBC - Java DataBase Connectivity - interfejs programowania opracowany w 1996 r. przez Sun Microsystems, umożliwiający niezależnym od platformy aplikacjom napisanym w języku Java porozumiewać się z bazami danych za pomocą języka SQL.
 \item DAO(ang. Data Access Object) - wzorzec projektowy wprowadzający rozdzielenie mechanizmu trwałości obiektów od reguł biznesowych.
 \item Service - Spring Service Beans - obiekty serwisowe wykonujące logikę biznesową.
\end{itemize}

Dla warstwy prezentacji wykorzystałem również wsparcie Spring. W 2008 roku została wydana przez społeczność Spring pierwsza wersja środowiska do tworzenia aplikacji desktopowych w JAVIE - SpringRCP - Rich Client Project. Niestety po wydaniu wersji 1.1 w 2009 roku projekt, który był na wyższym technologicznie poziomie niż NetBeans-RCP lub Eclipse-RCP z nieznanych powodów przestał być rozwijany\cite{spring_rcp_reference}. W roku 2011 jeden z głównych projektantów Spring-RCP postanowił go odświeżyć. W niecałe 6 miesięcy przepisano całe środowisko używając najnowszej metodologii pochodzącej z Spring 3. Nowa wersja projektu otrzymała nazwę Valkyrie-RCP\cite{valkyrie_reference}. W kolejnych iteracjach dodanych zostało wiele nowych funkcjonalności, w wyniku których po kolejnych 3 miesiącach powstało solidne narzędzie do tworzenia skomplikowanych aplikacji desktopowych w JAVIE. Jego podstawowe cechy to przejrzystość, prostota, stosowanie dobrych wzorców GUI.
Podstawowe elementy pakietu Valyrie-RCP, które wykorzystano do stworzenia aplikacji to\cite{valkyrie_reference}:
\begin{itemize}
 \item Widget - komponent GUI (tabelka, forma, widok, edytor-danych).
 \item DataProvider - fasada danych dla komponentów UI.
\end{itemize}

Rysunek \ref{img:rysunek_spring2} przedstawia architekturę systemu jaki został zaimplementowany.

\begin{figure}[!ht]
\centering
\includegraphics[scale=0.4]{images/spring-layers2}
\caption[Rysunek przedstawiający model architektury systemu]{Architektura systemu stworzonego dla pracy. Autor: Mateusz Urbanik}
\label{img:rysunek_spring2}
\end{figure}

Należy zaznaczyć, że model architektury spełnia założenia architekty trójwarstwowej:
\begin{table}[h]
 \caption{Moduły Spring jako elementy architektury trójwarstwowej}
 \small\tt
 \centering
 \vspace{0in}
 \begin{tabular}{|l|l|}
 \hline
 \textbf{Element architektury 3-warstwowej} & \textbf{Moduły systemu} \\ 
 \hline
 Dostęp do danych & H2-Database, JDBC, DAO \\
 \hline
 Logika biznesowa & Services \\
 \hline
 Prezentacja & Valkyrie-RCP \\
 \hline
 \end{tabular}
\end{table}

%---------------------------------------------------------------------------
\subsection{Architektura systemu ekspertowego}
\label{sec:architekturaSystemuEkspertowego}
Jedna z definicji systemu ekspertowego definiuje go jako system charakteryzujący się strukturą\cite{goluchowski_eskpertowe}:
\begin{itemize}
 \item 	Szkielet,
       \begin{itemize}
	 \item Interfejs użytkownika,
	 \item Edytor bazy wiedzy,
	 \item Systemu wnioskujący,
	 \item Mechanizm wyjaśniający,
       \end{itemize}
 \item Baza wiedzy,
 \item Dynamiczna baza danych.
\end{itemize}

Bardzo istotnym elementem systemu ekspertowego jest oddzielenie wiedzy dziedzinowej od reszty systemu. Takie podejście umożliwia usprawnienie działania systemu bez ingerencji w kod programu. W~skrócie wyjaśnię elementy architektury systemu ekspertowego.

\textbf{Interfejs użytkownika} umożliwia zadawanie pytań, udzielanie informacji systemowi oraz odbieranie od systemu odpowiedzi i wyjaśnień.
\textbf{Edytor bazy wiedzy} pozwala na modyfikację wiedzy zawartej w systemie, umożliwiając tym samym jego rozbudowę.
\textbf{Mechanizm wnioskowania} jest głównym składnikiem systemu ekspertowego wykonującym cały proces rozumowania w trakcie rozwiązywania problemu postawionego przez użytkownika.
\textbf{Mechanizm wyjaśniający} jest to jeden z elementów interfejsu pomiędzy systemem a użytkownikiem, który umożliwia użytkownikowi uzyskanie odpowiedzi dlaczego system udzielił takiej, a nie innej odpowiedzi, albo dlaczego system zadał użytkownikowi określone pytanie.
\textbf{Baza wiedzy} jest to deklaratywna postać wiedzy ekspertów z danej dziedziny zapisana za pomocą wybranego sposobu reprezentacji wiedzy, najczęściej reguł.
\textbf{Dynamiczna baza danych} jest pamięcią roboczą przechowującą pewne fakty wprowadzone w trakcie dialogu z użytkownikiem. Baza ta umożliwia odtworzenie sposobu wnioskowania systemu i przedstawienie go użytkownikowi za pomocą mechanizmu wyjaśniającego\cite{martyniuk_ekspertowe}.
Architekturę systemu regułowego przedstawia rysunek \ref{img:rysunek_expert}.
\begin{figure}[!ht]
\centering	
\includegraphics[scale=0.5]{images/expert-system}
\caption[Rysunek przedstawiający model architektury systemu ekspertowego]{Architektura systemu ekspertowego. Autor: Mateusz Urbanik}
\label{img:rysunek_expert}
\end{figure}

%---------------------------------------------------------------------------
\begin{comment}
\subsection{System ekpertowy w architekturze Spring}
\label{sec:systemEkpertowyArchitekturaSpring}

Na początku pracy postawiona została teza: realizacja systemu ekspertowego w architekturze spring spełniającego wymagania dotyczące aplikacji grupera oraz zawierającego optymalizację grupowania.

Popularne narzędzia do implementacji systemu ekspertowego to język Prolog, system Drools, Jess, CLIPS. Są to gotowe środowiska do budowania skomplikowanych systemów ekspertowych. Dla tych systemów wystarczy zdefiniować wiedzę w postaci faktów i reguł, a następnie zadać cele, aby otrzymać wynik. Nie zdecydowałem się jednak na użycie tych dedykowanych systemów z dwóch ważnych powodów. Po pierwsze w mojej pracy postanowiłem zrobić coś innowacyjnego, a mianowice chciałem zbudować swój autorski system ekspertowy od podstaw zarazem używając do jego budowy nowoczesnych narzędzi. Po drugie stworzenie chociaż zalążka takiego innowacyjnego systemu w najnowszej technologii będzie już sporym osiągnięciem. Spojrzenie na system ekspertowy z naciskiem na jego realizację w czystej JAVIE może przynieść ciekawe rezultaty. 

Ten sposób myślenia doprowadził mnie do punktu, w którym musiałem jasno określić, które elementy systemu ekspertowego będą zaimplementowane przez które elementy architektury Spring. I tak określiłem, że warunki dla spełnienia reguł mogą być zapisywane jako obiekty POJO w bazie danych. Same reguły mogą być zapisywane w bazie poprzez sygnatury metod do wykonania lub jako skrypty napisane w języku Groovy(obiektowy język skryptowy wzorowany na składni Javy). Dynamiczna baza wiedzy będzie zrealizowana za pomocą obiektów POJO zapisywanych w bazie danych. Interfejs użytkownika  powstanie przy użyciu środowiska ValkyrieRCP, oraz jego wzorca DataEditor, który umożliwi dodawanie, edycję, usuwanie a nawet filtrowanie faktów oraz uruchamianie algorytmu wnioskowania. System wnioskujący będzie zrealizowany w warstwie biznesowej/serwisowej, gdzie zostanie zaimplementowany mechanizm wnioskowania. Będzie on wybierał dla danych wejściowych zestaw reguł i warunków, następnie wywoła metody sprawdzające i wykona odpowiednie działania jeśli warunki są spełnione. Mechanizm wyjaśniający zaimplementowany zostanie również w warstwie biznesowej.
\end{comment}

%---------------------------------------------------------------------------
%---------------------------------------------------------------------------
\section{Model danych}
\label{sec:modelDanych}

W tym podrozdziale znajduje się opis w jaki sposób stworzony został model danych dla aplikacji.

%---------------------------------------------------------------------------
\subsection{Dane wejściowe NFZ}
\label{sec:daneWejscioweNFZ}
Jedynym źródłem danych wejściowych dla aplikacji były rozporządzenia NFZ opublikowane na oficjalnej witrynie internetowej Funduszu\cite{rozporzadzenia_nfz}. Fundusz opublikował ponad 30 wersji pliku parametryzującego w formacie MS-Excel. W pliku tym znajdują się arkusze kalkulacyjne, w których dane zostały zapisane w postaci tabel. Każda następna wersja pliku parametryzującego począwszy od wersji pierwszej dostarcza zaktualizowane dane(np. nowe kody rozponań, procedur). Z kolejnymi wydaniami pliku poprawione zostały drobne błędy związane z formatem zapisu danych. Aby zobrazować sposób w jaki dane zmieniały się przeanalizowano zapis w kolumnie ranga procedury. Ranga większa niż 2 była zapisywana początkowo jako napis '>2', zmieniono ten zapis na wartość '3'. Ze względu na fakt, iż wiele poprawiono wraz z wydaniami kolejnych wersji pliku parametryzującego postanowiono pracować na wersji 25, która jest jedną z najnowszych. 

Kolejne wersje pliku z danymi nie wnosiły żadnych zmian do modelu danych, który pozostawał dalej źle zdefiniowany. Dla przykładu arkusz o nazwie 'JPG' zawiera listę kodów JGP, gdzie jako wartości kolumn przyjęte zostały wartości oddziałów, a na przecięciu wiersza(kodu, kodu produktu, nazwy JGP) i kolumny(oddział) postawiony został krzyżyk(znak 'X'). Inny przykład źle zdefiniowanego modelu danych to arkusz, w którym zapisane są wartości punktowe dla kodów JGP. W arkuszu tym zostały przepisane jeszcze raz wszystkie kody JGP, kody produktu i nazwy grup, a obok podane zostały wartości punktowe. Jeszcze jednym przykładem nieprawidłowego zapisu danych jest zdefiniowanie w arkuszu 'procedury' redundantnych danych. Kody i nazwy procedur zostały zwielekrotnione, chociaż różniły się one jedynie kolumną 'typ listy'.

Plik parametryzujący dostarczał redundantne dane, zwielokrotnione wartości w wielu tabelach, nieokreślone zostały klucze podstawowe, obce, ani ograniczenia unikalności. Po wykonaniu dogłębnej analizy pliku wyciągnięto następujący wniosek: model danych dostarczany przez NFZ musi zostać przekonwertowany tak, aby nowy model danych zachowywał integralność. Przyjęto następującą strategię konwersji danych NFZ: dane z arkuszy kalukulacyjnych zostaną zaimportowane bezpośrednio do relacyjnej bazy danych, a następnie zostanie przeprowadzony proces normalizacji.

%---------------------------------------------------------------------------
\subsection{Relacyjna baza danych}
\label{sec:relacyjnaBazaDanych}

Wszystkie dane postanowiono zapisywać w relacyjnej bazie danych ze względu na nastepujące właściwości tego rozwiązania\cite{bazy_mimuw}:
\begin{itemize}
 \item Wszystkie wartości danych oparte są na prostych typach danych.
 \item Wszystkie dane w bazie relacyjnej przedstawiane są w formie dwuwymiarowych tabel.
 \item Po wprowadzeniu danych do bazy, możliwe jest porównywanie wartości z różnych kolumn, zazwyczaj również z różnych tabel, i scalanie wierszy, gdy pochodzące z nich wartości są zgodne. Umożliwia to wiązanie danych i wykonywanie stosunkowo złożonych operacji w granicach całej bazy danych.
 \item Bez względu na położenie wiersza tabeli wszystkie operacje wykonywane są w oparciu o algebrę relacji.
 \item Z braku możliwości identyfikacji wiersza przez jego pozycję pojawia się potrzeba obecności jednej lub więcej kolumn niepowtarzalnych w granicach całej tabeli, pozwalających odnaleźć konkretny wiersz. Kolumny te określa się jako klucz podstawowy tabeli.
\end{itemize}
Uniwersalność tego rozwiązania pozwala na zapis wszystkich typów danych, począwszy od typów prostych skończywszy na typach złożonych takich jak zserializowane obiekty JAVY, skrypty Groovy'iego lub pliki binarne. Taki sposób zapisu danych wpływa na elastyczność systemu.

Ważną decyzją projektową jest wybór konkretnego silnika bazy danych. Szukano rozwiązania, które pozwoli na zapis/odczyt danych z pliku bez potrzeby instalacji aplikacji serwera bazo-danowego w systemie operacyjnym. Tabela \ref{table_h2} przedstawia wybrane właściwości znanych silników bazodanowych(Źródło: \underline{\texttt{www.h2database.com}}[dostęp 4-04-2012]).

\begin{table}[h]
 \caption{Porównanie wybranych właściwości silników bazodanowych.}
 \tiny\tt
 \centering
 \vspace{0in}
 \begin{tabular}{|l|l|l|l|l|l|}
 \hline
  & \textbf{H2} & \textbf{Derby} & \textbf{HSQLDB} & \textbf{MySQL} & \textbf{PostgreSQL} \\
 \hline
 Pure Java & Yes & Yes & Yes & No & No \\
 \hline
 Memory Mode & Yes & Yes & Yes & No & No \\
 \hline
 Encrypted Database & Yes & Yes & Yes & No & No \\
 \hline
 ODBC Driver & Yes & No & No & Yes & Yes \\
 \hline
 Fulltext Search & Yes & No & No & Yes & Yes \\
 \hline
 Multi Version Concurrency & Yes & No & Yes & Yes & Yes \\
 \hline
 Footprint (jar/dll size) & ~1 MB & ~2 MB & ~1 MB & ~4 MB & ~6 MB \\
 \hline
 \end{tabular}
 \label{table_h2}
\end{table}

H2 to silnik bazodanowy spełniający postawione wymagania. Jest to rozwiązanie typu ,,open source'', którego autorem jest Thomas Mueller. Zaletą silnika H2 jest łatwość administracji bazy z poziomu przeglądarki(H2-Console). Implementacja JDBC-API pozwala na łatwą konfigurację w aplikacji. Brak potrzeby instalowania serwera bazy danych w systemie operacyjnym jest kolejnym atutem tego rozwiązania. Domyślnie silnik posiada funkcje CSVREAD i CSVWRITE, które pozwalają na import i~eksport danych z poziomu języka SQL. Dokumentacja do silnika jest przejrzysta oraz aktualizowana na bieżąco\cite{h2_reference}.

%---------------------------------------------------------------------------
\subsection{Normalizacja danych}
\label{sec:normalizacjaDanych}

Plik parametryzujący, w którym znajdują się wszystkie potrzebne dla aplikacji dane zaimportowano do bazy danych, a bazę poddano procesowi normalizacji. Każdy z arkuszy z pliku parametryzującego zapisano w osobnym pliku CSV. Każdy z plików CSV zaimportowano używając funkcji CSVREAD silnika H2 do odpowiadających im tabel tymczasowych w systemie bazy danych. Przykład importu tabeli przedstawia listing~\ref{sql_import_csv}.

\begin{lstlisting}[language=SQL,caption={Import listy kodów ICD9 z pliku CSV. Autor: Mateusz Urbanik},label=sql_import_csv]
CREATE TEMPORARY TABLE ICD9_TMP(
    LIST_CODE_ID VARCHAR(5) NOT NULL, -- Kod listy
    LIST_TYPE_ID CHAR(1) NOT NULL,    -- Typ listy
    CODE VARCHAR(7) NOT NULL,         -- Kod procedury ICD-9
    RANGE INT,                        -- Ranga procedury ICD-9
    NAME VARCHAR(255) NOT NULL        -- Nazwa procedury ICD-9
) AS SELECT LIST_CODE_ID, LIST_TYPE_ID, CODE, RANGE, NAME FROM CSVREAD('icd9.csv');
\end{lstlisting}

Następnym krokiem jest normalizacja bazy danych. Proces normalizacji bazy przeprowadzony został zgodnie z regułami postaci normalnych: 1PN, 2PN oraz 3PN\cite{bazy_mimuw}. Rozdzielono tabele, wprowadzono klucze publiczne i obce, usunięto zwielokrotnione wiersze. Listing \ref{sql_normalizacja_create} przedstawia skrypt tworzący nowe tabele.
\newpage
\begin{lstlisting}[language=SQL,caption={Normalizacja - tworzenie tabel dla listy kodów ICD9. Autor: Mateusz Urbanik},label=sql_normalizacja_create]
CREATE TABLE ICD_LIST_TYPE(
    ID CHAR(1) PRIMARY KEY,
    NAME VARCHAR(10) NOT NULL
);

CREATE TABLE ICD9(
    CODE VARCHAR(7) PRIMARY KEY,
    RANGE INT,
    NAME VARCHAR(255) NOT NULL
);

CREATE TABLE ICD9_LIST(
    CODE VARCHAR(5) PRIMARY KEY
);

CREATE TABLE ICD9_LIST_CODE(
    ICD9_CODE VARCHAR(7) NOT NULL,
    LIST_CODE VARCHAR(5) NOT NULL,
    LIST_TYPE CHAR(1) NOT NULL,
    PRIMARY KEY (ICD9_CODE, LIST_CODE),
    FOREIGN KEY(ICD9_CODE) REFERENCES ICD9(CODE),
    FOREIGN KEY(LIST_TYPE) REFERENCES ICD_LIST_TYPE(ID),
    FOREIGN KEY(LIST_CODE) REFERENCES ICD9_LIST(CODE)
);
\end{lstlisting}

Po utworzeniu znormalizowanego schematu danych skopiowano dane z tabel tymczasowych do docelowych pomijając zwielokrotnione wiersze. Listing \ref{sql_normalizacja_insert} przedstawia skrypt SQL wypełniający tabele docelowe dla listy kodów ICD9.

\begin{lstlisting}[language=SQL,caption={Normalizacja - wypełnianie tabel danymi dla listy kodów ICD9. Autor: Mateusz Urbanik},label=sql_normalizacja_insert]
INSERT INTO ICD_LIST_TYPE (ID, NAME) VALUES ('G', 'globalna');
INSERT INTO ICD_LIST_TYPE (ID, NAME) VALUES ('U', 'do sekcji');
INSERT INTO ICD_LIST_TYPE (ID, NAME) VALUES ('H', 'do grupy');
INSERT INTO ICD_LIST_TYPE (ID, NAME) VALUES ('N', 'negatywna');

INSERT INTO ICD9 (CODE, RANGE, NAME)
 SELECT DISTINCT CODE, RANGE, NAME FROM ICD9_TMP ORDER BY CODE;

INSERT INTO ICD9_LIST (CODE)
 SELECT DISTINCT LIST_CODE_ID FROM ICD9_TMP ORDER BY LIST_CODE_ID;

INSERT INTO ICD9_LIST_CODE (ICD9_CODE, LIST_CODE, LIST_TYPE)
 SELECT CODE, LIST_CODE_ID, LIST_TYPE_ID FROM ICD9_TMP ORDER BY CODE;
\end{lstlisting}

Spełniając założenia postaci normalnych bazy przeniesiono wszystkie potrzebne dane z arkuszy kalkulacyjnych do tabel bazy danych SQL. Wprowadzono relacje poprzez ustalenie kluczy publicznych oraz obcych. Dla kolumn wymagających wprowadzenia klucza ograniczenia unikalności, zostało ono dodane. W celu przyspieszenia działania instrukcji SELECT dodano indeksy do kolumn.
 
W wyniku procesu normalizacji i wstępnej optymalizacji schematu bazy, utworzonych zostało 16 plików SQL. Skrypty importują dane do tabel tymczasowych, następnie wykonują instrukcje tworzące schemat, który spełnia warunki spójności oraz zabiega anomaliom bazo-danowym. Ostatnim krokiem jest przepisanie danych, w którym pomijane są zwielokrotnione wpisy. Skrypty zostały uporządkowane w kolejności, w której muszą zostać wykonane, aby stworzyć model wypełniając go danymi z pliku opublikowanego przez NFZ. Tabela \ref{table_csv_sql} zawiera wszystkie wyróżnione tabele oraz powiązane z nimi pliki z danymi.

\begin{table}[h]
   \caption{Pliki oraz tabele  - model danych. Autor: Mateusz Urbanik}
   \tiny\tt
   \centering
   \vspace{0in}
   \begin{tabular}{|c|l|l|l|}
      \hline
      \textbf{arkusz} & \textbf{plik CSV} & \textbf{plik SQL} & \textbf{Nazwa tabeli} \\
      \hline
      wersja JGP & - & - & - \\
      \hline
      ograniczenie pobytu & - & 1\_time\_unit.sql & TIME\_UNIT \\
      \hline
      ograniczenie wieku & - & 2\_age\_limit.sql & AGE\_LIMIT \\
      \hline
      ograniczenie pobytu & - & 3\_hospital\_limit.sql & HOSPITAL\_LIMIT \\
      \hline
      JGP & jgp\_department.csv & 4\_department.sql & DEPARTMENT \\
      \hline
      ograniczenie trybu przyjęcia & - & 5\_income\_mode\_limit.sql & INCOME\_MODE\_LIMIT \\
      \hline
      ograniczenie trybu wypisu & - & 6\_outcome\_mode\_limit.sql & OUTCOME\_MODE\_LIMIT \\
      \hline
      - & - & 7\_patient.sql & PATIENT \\
      \hline
      wykaz specjalności komórek & specialization\_unit.csv & 8\_specialization\_unit.sql & SPECIALIZATION\_UNIT \\
      \hline
      wykaz specjalności komórek & specialization\_unit.csv & 8\_specialization\_unit.sql & SPECIALIZATION\_UNIT\_EXCLUDE\_SERVICE \\
      \hline
      listy procedur & icd9.csv & 9\_icd\_list\_type.sql & ICD\_LIST\_TYPE \\
      \hline
      listy procedur & icd9.csv & 10\_icd9.sql & ICD9 \\
      \hline
      listy procedur & icd9.csv & 10\_icd9.sql & ICD9\_LIST \\
      \hline
      listy procedur & icd9.csv & 10\_icd9.sql & ICD9\_LIST\_CODE \\
      \hline
      listy rozpoznań & icd10.csv & 11\_icd10.sql & ICD10 \\
      \hline
      listy rozpoznań & icd10.csv & 11\_icd10.sql & ICD10 \\
      \hline
      listy rozpoznań & icd10.csv & 11\_icd10.sql & ICD10\_LIST\_CODE \\
      \hline
      zakresy JGP & jgp.csv & 12\_jgp.sql & JGP \\
      \hline
      zakresy JGP & jgp.csv & 12\_jgp.sql & JGP\_POINT\_VALUE \\
      \hline
      mechanizm osobodni & jgp\_hospital.csv & 13\_jgp\_hospital.sql & JGP\_HOSPITAL \\
      \hline
      JGP & jgp\_department.csv & 14\_jgp\_department.sql & JGP\_DEPARTMENT \\
      \hline
      parametry JGP & jgp\_parameter.csv & 15\_jgp\_parameter.sql & JGP\_PARAMETER \\
      \hline
   \end{tabular}
 \label{table_csv_sql}
\end{table}



%---------------------------------------------------------------------------
\newpage
\subsection{Diagram ERD}
\label{sec:diagramERD}

W wyniku procesu dogłębnej analizy modelu danych, a następnie jego normalizacji powstał spójny oraz odporny na anomalie bazodanowe schemat. Logiczną strukturę wynikowego modelu ilustruje rysunek \ref{img:diagram_erd} przedstawiający diagram ERD(Entity Relationship Diagram).

\begin{figure}[!ht]
\centering
\includegraphics[scale=0.31]{images/erd}
\caption[Diagram ERD]{Diagram ERD. Autor: Mateusz Urbanik}
\label{img:diagram_erd}
\end{figure}

%---------------------------------------------------------------------------
%---------------------------------------------------------------------------
\section{Gruper NFZ}
\label{sec:gruperNFZ}

W tym podrozdziale znajduje się skrócony opis działania algorytmu ,,Grupera''. Kompletny opis algorytmu został przedstawiony w dokumencie ,,Informacje do przygotowania aplikacji grupera na potrzeby szpitalnych systemów informatycznych umożliwiającego kwalifikację rekordu pacjenta do właściwej grupy systemu Jednorodnych Grup Pacjentów'' opublikowanym przez NFZ\cite{algorytm_grupera}. Zawiera on słowny opis algorytmu oraz schematy blokowe, które są dalekie do standardów takich jak na przykład język UML. Należy w tym miejscu zaznaczyć, że specyfikacja udostępniona przez Fundusz jest jedynym źródłem wiedzy o zasadach kierujących algorytmem.
% Poznanie algorytmu grupera, zrozumienie zasad jakimi się on kieruje było kluczowym punktem pracy.

%---------------------------------------------------------------------------
\subsection{Zasady i logika grupowania}
\label{sec:zasadyLogikaGrupowania}
Wynikiem działania algorytmu ,,Grupera'' jest grupa JGP, która spełnia określony zestaw warunków. Lista grup JGP oraz zestaw warunków jest wyznaczany na podstawie danych wejściowych opisujących hospitalizację pacjetna. Algorytm w pierwszym kroku(w dużym uproszczeniu) wyznacza grupy JGP dla danych epizodu. Po wyznaczeniu listy grup następuje badanie tzw. mechanizmem przeliczania. Jeśli dana grupa zostanie zakwalifikowana do zmiany wartości punktowej zostaje ona zmieniona zgodnie z algorytmem mechanizmu przeliczania\cite{algorytm_grupera}. Diagram aktywności zilustrowany na rysunku \ref{img:diagram_activity_gruper} przedstawia działanie algorytmu ,,Grupera''.

\begin{figure}[!ht]
\centering
\includegraphics[scale=0.4]{images/activity-gruper}
\caption[Diagram aktywności]{Diagram aktywności - algorytm ,,Grupera''. Autor: Mateusz Urbanik}
\label{img:diagram_activity_gruper}
\end{figure}

%---------------------------------------------------------------------------
\newpage
\subsection{Proces wyznaczania grupy systemu JGP}
\label{sec:procesWyznaczaniaGrupySystemuJGP}
Proces wyznaczania grupy można podzielić na dwie główne gałęzie. Pierwsza gałąź(tzw. ścieżka 'ICD-9') wyznacza zbiór zakwalifikowanych grup wraz z warunkami do sprawdzenia na podstawie procedur znaczących. Druga ścieżka 'ICD-10' wyznacza analogicznie zbiór zakwalifikowanych grup, ale na podstawie rozpoznań zasadniczych. Ścieżka dla rozpoznań jest wybierana przez program jeśli spełniony jest jeden z następujących warunków:
\begin{itemize}\itemsep2pt
\item w danych epizodu nie występuje żadna procedura
\item ranga procedury <= 2 i czas hospitalizacji > 1 dnia
\end{itemize}
Jeśli wymienione wyżej warunki nie są spełnione wybierana jest ścieżka ICD-9\cite{algorytm_grupera}.

Po wybraniu ścieżki wyznaczana jest lista kodów JGP wraz z zestawem warunków do przetestowania. Każda grupa, która spełnia wszystkie warunki jest zapisywana na liście wybranych grup. Proces wyznaczania grupy ilustruje diagram aktywnyści przedstawiony na rysunku~\ref{img:diagram_activity_jgp}.
\newpage
\begin{figure}[!ht]
\centering
\includegraphics[scale=0.4]{images/activity-jgp} 
\caption[Diagram aktywności]{Diagram aktywności - wyznaczanie grupy JGP. Autor: Mateusz Urbanik}
\label{img:diagram_activity_jgp}
\end{figure}

%---------------------------------------------------------------------------
\subsection{Warunki kierunkowe}
\label{sec:warunkiKierunkowe}
Warunki kierunkowe są to wymagania decydujące o przebiegu grupowania. Każdy z warunków został oznaczony literą alfabetu. Ich dokładny opis znajduje się w dokumencie NFZ definiującym algorytm grupera\cite{algorytm_grupera}. Procedura badająca warunki JGP w pierwszej kolejności sprawdza czy spełnione są warunki kierunkowe, ponieważ mają one największy wpływ na proces grupowania.

Sposób w jaki zdecydowano się zaimplementować warunki kierunkowe przedstawiają poniższe założenia:
\begin{itemize}
\item Każdy z warunków zapisany jest w bazie pojedynczą literą alfabetu (A-Z).
\item Każda z liter alfabetu zostanie zmapowana na odpowiednią wartość enuma 'Condition', który zawiera wszystkie sygnatury warunków kierunkowych.
\item Zdefiniowanych zostało 26 klas nazwanych dużymi literami alfabetu rozszerzających bazową klasę AbstractChecker oraz implementujących metodę \mbox{\textit{public abstract boolean checkCondition(Stay stay, JGPParameter parameter, List<Reason> reasons);}}
\item Utworzona została enum-mapa, w której kluczem jest wartość enum(sygnatura warunku kierunkowego), a wartością obiekt(singleton) przypisanego do niej 'sprawdzacza' warunków.
\end{itemize}
Zaletą takiego rozwiązania jest możliwość implementacji generycznego silnika do sprawdzania warunków kierunkowych. Zdefiniowano zatem algorytm działania takiego silnika. Dla każdego wybranego obiektu klasy \mbox{JGPParameter}(warunki które mają zostać sprawdzone) na podstawie odczytanej sygnatury warunku kierunkowego zostaje uruchamiany odpowiedni kawałek kodu JAVY sprawdzający poprawność warunków dla pobytu. Listing \ref{java_check} przedstawia kawałek kodu JAVY realizujący tą funkcjonalność.

\begin{lstlisting}[language=Java,caption={Metoda sprawdzająca warunki kierunkowe. Autor: Mateusz Urbanik},label=java_check]
private boolean checkDirectional(Stay stay, JGPParameter parameter, List<Reason> reasons) {
  //pobranie sygnatury(A-Z) warunku ktory powinien zostac sprawdzony
  Condition condition = parameter.getCondition();
  //pobranie checkera dla warunku z mapy
  AbstractChecker checker = conditionsMap.get(condition);
  Assert.notNull(checker, "not implemented checker for condition: " + condition);
  //uruchomienie odpowiedniego checkera i zwrocenie wyniku badania warunku
  return checker.checkCondition(stay, parameter, reasons);
}
\end{lstlisting}

Rozwiązanie to pozwoliło na zapis skomplikowanych warunków w bazie danych. Jednak zapisywanie sygnatury obiektu 'sprawdzacza warunków', sprawia że nadal ważna część logiki jest zapisana w kodzie. O rozwinięciu tego rozwiązania  i o sposobie zapisu całej logiki ,,checkera'' do bazy danych napisano więcej w rozdziale \ref{sec:systemEkspertowy}.

%---------------------------------------------------------------------------
\subsection{Mechanizm osobodni}
\label{sec:mechanizmOsobodni}
Wyznacznikiem kosztów leczenia pacjenta nie może być tylko wartość punktowa wyliczona na podstawie przypisanej do pacjenta grupy systemu JGP. Ważnym czynnikiem generującym koszty leczenia jest czas pobytu w szpitalu. Pożywienie, wymiana pościeli, podawane leki i inne koszty generowane są przez pacjenta w każdym dniu jego pobytu na oddziale. Dlatego do algorytmu ,,Grupera'' musiał zostać wprowadzony mechanizm osobodni\cite{szkoleniaJGP}. Na początku algorytm bada czy można zastosować mechanizm przeliczania, co obrazuje diagram aktywności \ref{img:diagram_activity_maday}.

\begin{figure}[!ht]
\centering
\includegraphics[scale=0.5]{images/activity-manday}
\caption[Diagram aktywności]{Diagram aktywności - mechanizm osobodni. Autor: Mateusz Urbanik}
\label{img:diagram_activity_maday}
\end{figure}

Jeśli wyznaczona grupa JGP dla epizodu nie kwalifikuje się do zmiany punktowej, pozostawiamy wartość wyznaczoną przez algorytm bez zmian. Natomiast jeśli zachodzi potrzeba zmiany wartości punktowej stosujemy następujący algorytm dla 2 przypadków:
\begin{itemize}
\item Jeśli czas hospitalizacji jest mniejszy od 2 dni ustalamy nową wartość punktową ustalaną na podstawie wartości 'poniżej 2 dni' z obiektu klasy JGPHospital. Należy zaznaczyć, że z powyższego mechanizmu są wyłączone zgony, świadczenia rozliczane w ramach umów z zakresu ,,zespół opieki dziennej'' oraz ,,zespół chirurgii jednego dnia''.
\item Jeśli czas trwania episodu jest większy lub równy 2 dni i jest większy od limitu ustalonego na podstawie wartości ,,dni'' z obiektu klasy JGPHospital. To wartość JGP jest ustalana poprzez powiększenie aktualnej wartości JGP o wynik iloczynu dni powyżej limitu z wartością punktową określoną przez wartość "powyżej".

NOWA WARTOŚĆ PUNKTOWA = STARA WARTOŚĆ + RÓŻNICA DNI PONAD LIMIT * WARTOŚĆ Z KOLUMNY ,,POWYŻEJ LIMITU''
\end{itemize} 

%---------------------------------------------------------------------------
%---------------------------------------------------------------------------
\section{System ekspertowy}
\label{sec:systemEkspertowy}

Stworzenie systemu ekspertowego jest zadaniem trudnym\cite{mulawka_ekspertowe}. Zdecydowano się na podejście ewolucyjne, w którym system w trakcie rozwijania go będzie dążył do rozwiązania spełniającego założenia systemu ekspertowego. Zbliżanie się w kierunku systemu ekspertowego oznacza, że nie powstaje on od razu, ale w wyniku przekształceń pewnych części systemu. Zmiany zachodzące w implementacji sprawiają, że staje się ona coraz bardziej generyczna i uniwersalna. Zaletą takiego podejścia jest powolne, jasne i zrozumiałe wprowadzanie zmian w kodzie źrodłowym systemu, w trakcie których system ciągle spełnia stawiane mu wymagania. Wadą jest potrzeba dużego nakładu pracy, aby uzyskać w efekcie generyczny system ekspertowy\cite{zielonogorski_ekspertowe}.

\subsection{Klasyfikacja}
\label{sec:klasyfikacjaSystemuEkspertowego}
W tym podrozdziale zakwalifikowano stworzony system do odpowiednich klas systemów ekspertowych:
\begin{itemize}
 \item Ze względu na możliwość ingerencji człowieka w produkowane przez system rozwiązanie jest to tzw. system doradczy \textendash{} podpowiada rozwiązanie pomagając podjąć decyzję użytkownikowi \textendash{} prezentuje rozwiązanie jakiegoś problemu, ale do użytkownika należy jego ocena, oraz to czy je zaakceptuje, czy odrzuci\cite{goluchowski_eskpertowe}.
 \item Ze względu na złożoność jest to system płytki - tzn. taki który korzysta tylko z informacji zgromadzonych w bazie wiedzy\cite{goluchowski_eskpertowe}.
 \item Ze względu na dane otrzymywane na wyjściu\cite{mulawka_ekspertowe} jest to system:
   \begin{itemize}
    \item Diagnozy \textendash{} ocena istniejącego stanu na podstawie posiadanych danych,
    \item Planowania \textendash{} opis stanu, do którego należy dążyć.
   \end{itemize}
 \item Ze względu na rodzaj przetwarzanej informacji jest to system z wiedzą pewną\cite{mulawka_ekspertowe}.
\end{itemize}

\subsection{Baza wiedzy}
\label{sec:bazaWiedzy}
Najważniejszym z punktu widzenia systemu ekspertowego jest podział wiedzy na bazę faktów oraz bazę reguł\cite{zielonogorski_ekspertowe}. Logiczne wyróżnienie faktów dla systemu grupera JGP, pozwoli spojrzeć na aplikację ,,Grupera'' jak na system ekpertowy. Wyróżniono zatem fakty: rekord pacjenta, płeć, oddział, rozpoznanie(ICD-10), procedura(ICD-9), tryb przyjęcia, tryb wypisu, Jednorodna Grupa Pacjentów, wartości punktowe dla JGP, czas hospitalizacji.

Zdefiniowano nastepujące warunki: warunki kierunkowe, warunek na rozpoznanie główne, warunek na rozpoznanie dodatkowe, warunek na procedurę zasadniczą, warunki na procedury dodakowe, warunek na procedury i rozpoznania wykluczające się, ograniczenie na czas hospitalizacji, ograniczenie na wiek.
Spełnienie wyżej wymienionych warunków powoduje wykonanie akcji: zapisanie kodu JGP na listę wybranych kodów. Reguła ta stosowana jest do wyznaczania listy zaakceptowanych grup JGP.

Proces wyznaczania nowych wartości mechanizmem osobodni, gdzie sprawdzane są warunki na czas hospitalizacji, a w przypadku ich spełnienia wyznaczana jest nowa wartość punktowa jest kolejnym przykładem reguły w systemie ekspertowym.

\subsection{Maszyna wnioskująca}
\label{sec:maszynaWnioskujaca}
Realizacja maszyny wnioskującej w języku JAVA odbywa się na poziomie warstwy biznesowej w technologii Spring. Klasa serwisowa przeprowadzająca proces wnioskowania zgodnie z wymaganiami systemu JGP to JGPService(Listing \ref{java_jgp_service}).

\begin{lstlisting}[language=Java,caption={Serwis JGP - realizacja maszyny wnioskującej. Autor: Mateusz Urbanik},label=java_jgp_service]
public interface JGPService {
    public List<JGP> findJGP(final JGPFilter filter);

    public JGPGroupResult group(Episode episode);

    public JGPGroupResult doByProcedures(Episode episode);

    public JGPGroupResult doByRecognitions(Episode episode);

    public void resolveResultsByJGP(Stay stay, List<JGPParameter> parameters, JGPGroupResult jgpGroupResult);

    public void recountManDay(Episode episode, List<JGPResult> jgpResultList);
}
\end{lstlisting}

Metoda \textit{,,group''} na podstawie danych epizodu wybiera główną ścieżkę poszukiwania. Istnieją dwie opcje: według rozpoznania głównego lub według procedur zasadniczych. Metody \textit{,,doByRecognitions''} oraz \textit{,,doByProcedures''} wyznaczają zbiór warunków z charakterystyki JGP do przeanalizowania. Metoda \textit{,,resolveResultsByJGP''} wykonuje 2 zasadnicze fazy procesu wnioskowania:
\begin{enumerate}\itemsep1pt
 \item Sprawdza warunki kierunkowe, warunek na płeć, warunek na tryb przyjęcia i wypisu, warunek na leczenie w danym oddziale, warunki na współisteniejące i zasadnicze kody ICD , warunki na kody ICD wykluczające się(negatywne).
 \item Jeśli zostaną one spełnione to zapisuje kod JGP na liście zaakceptowanych kodów.
\end{enumerate}

Podusmowując, maszyna wnioskująca ma za zadanie poszukiwanie optymalnej grupy JGP. Realizuje ona proces wnioskowania w następujących krokach. Przyjmuje zbiór faktów wejściowych w postaci obiektu klasy Episode. Następnie wybiera ścieżkę(ICD9 lub ICD10) według której mają zostać ustalone parametry JGP. Po ustaleniu ścieżki poszukiwania, wyciągane są z bazy warunki(w postaci obiektu klasy JGPParameter), które mają zostać przetestowane. Następuje sprawdzenie warunków i w przypadku spełnienia wszystkich grupa zostaje zapisana na liście zaakceptowanych grup.


%---------------------------------------------------------------------------
%---------------------------------------------------------------------------
\section{Optymalizacja JGP}
\label{sec:optymalizacjaJGP}

W tym podrozdziale opisano zagadanienie optymalizacji kosztów leczenia pacjenta. W podrozdziale~\ref{sec:maszynaWnioskujaca} ustalono, że maszyna wnioskująca wybiera zbiór parametrów do przetestowania. Następnie dla grup JGP spełniających warunki, zapisuje je na liście zaakceptowanych grup.
Grupy niespełniające warunki są odrzucane. Tę utraconą informację postanowiono wykorzystać. Zapostulowano dodanie następujących funkcjonalności do algorytmu:
\begin{enumerate}
 \item Zapisać testowane grupy JGP odrzucone przez algorytm na oddzielną listę(grupy niezaakceptowane).
 \item Dla każdej niezaakceptowanej grupy JGP obliczyć jaką wartość punktową przyjęłaby, gdyby została zaakceptowana przez algorytm.
 \item Stworzyć listę powodów odrzucenia grupy i zapisać ją dla każdej niezaakceptowanej grupy.
\end{enumerate}

Rozszerzono implementację algorytmu o powyższe funkcjonalności. Do obiektu klasy \textit{JGPResult} dodano listę niezaakceptowanych grup. Każda niezaakceptwana grupa będzie posiadać dodatkowo listę powodów niezaakceptowania, czyli listę obiektów klasy Reason. 
Taka implementacja sprawia że, system pamięta informację jakie dokładnie warunki były testowane dla konkretnego rozwiązania oraz dlaczego nie zostały one zaakceptowane. Hierarchię klas powodów niezaakceptowania przedstawia rysunek \ref{img:diagram_class_reason}.

\begin{figure}[!ht]
\centering
\includegraphics[scale=0.4]{images/reason-classes2}
\caption[Diagram aktywności]{Diagram klas - hierarchia powodów niezaakceptowania. Autor: Mateusz Urbanik}
\label{img:diagram_class_reason}
\end{figure}

\newpage
Rozszerzono listę argumentów każdej z metod sprawdzających warunki o argument \mbox{\textit{,,List<Reason> reasons''}}(lista powodów niezaakceptowania). Powodem niezaakceptowania grupy systemu JGP jest niespełnienie konkretnego warunku.
W przypadku niezaakceptowania warunku tworzona jest odpowiednia instancja obiektu \textit{Reason} i jest ona  dodawana do listy powodów niezaakceptowania. Przykładowa implementacja tej funkcjonalności jest przedstawiona na listingu \ref{java_check_hosp_limit}.

\begin{lstlisting}[language=Java,caption={Metoda sprawdzająca warunek na czas hospitalizacji. Autor: Mateusz Urbanik},label=java_check_hosp_limit]
protected boolean checkHospitalLimit(Stay stay, HospitalLimit hospLimit, List<Reason> reasons) {
     if (hospLimit != null) {
         int time = stay.getEpisode().hospitalTime(hospLimit.getTimeUnit());
         boolean result = hospLimit.test(time);
         if (!result) {
             reasons.add(new HospitalReason(hospLimit, condition()));
         }
         return result;
     }
     return true;
}
\end{lstlisting}

Metoda \textit{,,checkHospitalLimit''} realizuje dopisanie powodu niezaakcpetowania grupy JGP w przypadku, gdy test sprawdzający czas hospitalizacji epizodu będzie miał wynik negatywny. Przykładem zapisu powodu niezaakceptowania jest przypadek, gdy czas hospitalizacji dla grupy JGP musi być większy niż 7 dni, a zdefiniowany czas leczenia jest równy 5 dni.

Takie podejście pozwala na stworzenie funkocjonalności, która będzie podpowiadać lekarzowi jakie może uzyskać grupy JGP z większą wartości punktową, czyli jak zmaksymalizować koszty leczenia pacjenta. Program podpowiadałby również jakie konkretnie warunki musi spełnić hospitalizacja, aby zaliczyć ją do droższej grupy (wybranej przez lekarza).

Zaimplementowanie mechanizmu ,,Reasons - powody niezaakceptowania do grupy'' odpowiada w systemie ekspertowym modułowi objaśniająco-wyjąśniającemu(ang. Explanation Facility). Jest to część systemu odpowiedzialna za wyprowadzanie na zewnątrz wniosków systemu. Moduł ten daje użytkownikowi radę, sugestię, ale nie podejmuje decyzji\cite{martyniuk_ekspertowe}. Należy zwrócić uwagę na fakt, iż zapis historii niespełnionych warunków pozwala odtworzyć część ścieżki poszukiwania rozwiązania.

%---------------------------------------------------------------------------

\section{Porównanie z innymi istniejącymi rozwiązaniami}
\label{sec:porownanieZinnymiRozwiazaniami}

Przeanalizowano konkurencyjne rozwiązania na rynku. Najpopularniejsze aplikacje, postanowiono zainstalować, uruchomić i porównać z rozwiązaniem stworzonym na potrzeby pracy. %A oto listing systemów medycznych zawierających algorytm grupera:

\textbf{MP-System} - złożony system do kompleksowej obsługi szpitali. Już w trakcie instalacji wersji demo pojawiły się błędy dotyczące starej wersji serwera MS-SQL(dostarczanej wraz z aplikacją). Niestety nawet po kilku-godzinnych trudach nie udało się zainstalować i uruchomić aplikacji. Skomplikowany proces instalacji uniemożliwił zainstalowanie systemu Medycyny Praktycznej.

\textbf{Internetowy OPTYMALIZATOR JGP} CompuGroup Medical Polska Lublin - wersja demo dostępna pod adresem http://jgp.uhc.com.pl/jgpgrouper(dostęp 12.06.2012).
Aplikacja umożliwia przeporawdzenie pięciu darmowych grupowań dziennie. Powyżej tej liczby jest aplikacja jest płatna. Według statystyk zamieszonych na stronie przy użyciu aplikacji przeprowadzono 4mln grupowań dla kilkuset szpitali.%Jest to rozwiązanie wysokiej jakości, ale płatne.
System ten posiada również moduł do optymalizacji, z którego korzysta 58\% użytkowników(pozostałe 42\% korzysta z samego algorytmu grupowania).

\textbf{SMARTGRUPER} - http://smartgruper.pl(dostęp 12.06.2012). Jest to komercyjne rozwiązanie, nie posiadające wersji demo. Z opisu na stronie projektu wynika, że jest to oprogramowanie, które posiada wiele funkcjonalności poza standardowym ,,Gruperem'', na przykład moduł optymalizacji JGP, zapisywanie grupowań. Dostępne są 3 wersje produktu:
\begin{itemize}
 \item Wersja On-line. Dostęp indywidualny oraz dla małych placówek medycznych. Dostęp do systemu Smart Gruper on-line.
 \item Wersja Professional. Rozwiązanie dla szpitala bez integracji. Instalacja na serwerze w szpitalu.
 \item Wersja Enterprise. Rozwiązanie dla szpitala z integracją. Instalacja na serwerze w szpitalu oraz integracja z istniejącym systemem informatycznym szpitala.
\end{itemize}

Każdy z powyżej wymienionych systemów jest rozwiązaniem komercyjnym. System stworzony w ramach tej pracy jest aktualnie jedynym rozwiązaniem typu ,,open source'' na rynku oraz jedynym, które wnosi możliwości jakie daje profesjonalny system ekspertowy.

Przyszłością systemów do rozliczania kosztów leczenia pacjenta jest powstający system \mbox{Euro-DRG~7}, który docelowo stanowiłby jednolity system JGP dla całej Unii Europejskiej(http://www.eurodrg.eu).
NFZ stał się partnerem systemu Euro-DRG i aktywnie uczestniczy w jego tworzeniu. Być może stworzenie platformy systemu ekspertowego dla systemu Euro-DRG byłoby dobrym kierunkiem rozwoju tego typu rozwiązania. Jako projekt ,,open source'' posiadałby on swoich aktywnych uczestników nie tylko w Polsce, ale w całej Unii Europejskiej. Niestety aktualnie nie określona została jeszcze specyfikacja systemu. 
\begin{comment}
Problematyczne jest podejście wszystkich firm startujących w przetargach, ponieważ każda z nich chce mieć monopol na cały obszar zastosowania systemu. Jest to w mojej opinii poważna przeszkoda w tworzeniu najlepszej jakości oprogramowania, ponieważ w jej wyniku powstają systemy skomplikowane, ktorych używa się tylko dlatego, że nie ma innej alternatywy. Podobnie jak system operacyjny Linux jest ciągle ,,konkurencją'' dla systemu Winodws, tak samo oprogramowanie openSource w każdej dziedzinie będzie uczestniczyć w nieustającym wyścigu z oprogramowaniem komercyjnym. 
\end{comment}
%---------------------------------------------------------------------------

\section{Wykorzystane biblioteki}
\label{sec:wykorzystaneBiblioteki}
Tabela \ref{tab:tools} zawiera listę narzędzi wykorzystanych do stworzenia systemu.
\begin{table}[h]
 \caption{Wykorzystane narzędzia}
 \small\tt
 \centering
 \vspace{0in}
 \begin{tabular}{|l|p{7cm}|}
 \hline
 GIT & repozytorium kod źródłowy (https://github.com/urbanq/mgr) \\
 \hline
 commons-collections-3.2.1 & Pomocnicze metody dla operacji na kolekcjach \\
 \hline
 aspectj-1.6.10 & Aspekty dla JAVY \\
 \hline
 h2database-1.3.166 & Silnik bazodanowy H2 \\
 \hline
 junit-4.10 & Środowisko testowe \\
 \hline
 jgoodies-2.2.0 & GUI narzędzia pomocnicze \\
 \hline
 jodatime-2.1 & Najlepsza biblioteka do wykonywania obliczeń datowo/czasowych \\
 \hline
 spring-framework-3.0.5 & Spring framework z komponentami Core, JDBC, Security, Beans, Aspects, ORM, Test, TX \\
 \hline
 valkyriercp-1.0 & Biblioteka pozwalająca na wysokopozimowe programowanie GUI \\
 \hline
 \end{tabular}
 \label{tab:tools}
\end{table}

%---------------------------------------------------------------------------

\section{Jakość oprogramowania}
\label{sec:jakoscOprogramowania}

Zestawienie wartości liczbowych charakteryzujące oprogramowanie jest nazywane metryką oprogramowania. Dla projektu wykonanego w ramach pracy obliczyłem następujące wartości metryk chrakteryzujące jego wielkość:
\begin{itemize}
 \item 7009 LOC(Lines of code) - ilość linijek kodu źródłowego
 \item 191 C(Classes) - ilość zdefiniowanych klas JAVY.
\end{itemize}

Następne zestawienie metryk charakteryzujące jakość napisanego kodu to MOOD - Metrics for Object-Oriented Design. Metryki te służą do wyrażenia w procentach cech charakteryzujących dla programowania obiektowego:
\begin{itemize}
 \item 78,26\% AHF - Attribute Hiding Factor - hermetyzacja atrybutów
 \item 23,55\% MHF - Method Hiding Factor - hermetyzacja metod
 \item 71,49\% AIF - Attribute Inheritance Factor - dziedziczenie atrybutów
 \item 32,29\% MIF - Method Inheritance Factor - dziedziczenie metod
 \item 5,47\% CF - Coupling Factor - przekazywanie komunikatów 
 \item 57,21\% PF - polimorfizm
\end{itemize}


\chapter{Prezentacja działania aplikacji}
\label{cha:prezentacja}

W tej części pracy znajduje się opis aplikacji widzianej oczami użytkownika. W rozdziale~\ref{sec:interfejsUzytkownika} został
przedstawiony interfejs graficzny użytkownika. Następnie rozdział~\ref{sec:praktyczneUzycieSystemu} prezentuje działanie aplikacji wykorzystując przykładowe dane hospitalizacji. W rozdziale tym znajduje się opis praktycznego użycia systemu dla realnych danych. Sekcja~\ref{sec:testyIntegracyjne} zawiera przykładowy test integracyjny jednej ze ścieżek poszukiwania rozwiązania.

%---------------------------------------------------------------------------

\section{Interfejs użytkownika}
\label{sec:interfejsUzytkownika}
Aplikacja posiada standardowo okienkowy interfejs użytkownika. Skórka aplikacji jest ustawiona na przyjazny ,,PlasticXPLookAndFeel''.

Wszystkie funkcjonalności, takie jak dodawanie, usuwanie, filtrowanie oraz cały widoczny układ komponentów graficznych dostarcza framework Valkyrie-RCP. Realizuje je klasa ,,DataEditor'', która jest komponentem graficznym(Widgetem). Do obiektu klasy ,,DataEditor'' wstrzykiwane są JavaBean'y: \textbf{DataProvider} - fasada danych dla interfejsu użytkownika, \textbf{DetailForm} - forma do dodawania/usuwania danych, \textbf{FilterForm} forma dla dodatkowego filtrowania oraz \textbf{TableWidget} - obiekt definiujący wyświetlaną tabelkę danych\cite{valkyrie_reference}.
Rysunek \ref{img:patient} ilutruje widok bazy pacjetnów, który jest oparty na szablonie ,,DataEditor''.

\begin{figure}[!ht]
\centering
\includegraphics[scale=0.4]{images/patient} 
\caption[Widok bazy pacjentów]{Widok bazy pacjentów.}
\label{img:patient}
\end{figure}

W oparciu o wzorzec ,,DataEditor'' zbudowane są następujące widoki: katalog kodów \mbox{ICD-9}, \mbox{ICD-10}, JGP. Z tą różnicą w~stosunku do widoku bazy pacjentów, że zablokowana jest dla nich możliwość dodawania, edycji i~usuwania.

%---------------------------------------------------------------------------

\section{Praktyczne użycie systemu}
\label{sec:praktyczneUzycieSystemu}
Aby zilustrować działanie aplikacji przedstawiono w~tym podrozdziale konkretny i~realny przypadek użycia aplikacji. Zostanie wykorzystany również mechanizm dzięki, któremu lekarz ma możliwość optymalizacji kosztów leczenia pacjenta.

Zdefiniowano fakty wejściowe. Leczony pacjent urodzony w~1937 roku ma 75 lat, jest mężczyzną, przyjęty był na okres 7 dni w~trybie przyjęcia ,,Przyjęcie planowe na podstawie skierowania'' oraz wypisany trybie ,,Zakończenie procesu terapeutycznego lub diagnostycznego''. Typ hospitalizacji ustalono na ,,Hospitalizacja zwykła''. Rysunek \ref{img:gruper1} przedstawia uzupełniony powyższymi danymi widok ,,Grupera''. 
Następnie należy zdefiniować pobyt. Do tego celu służy okno dialowe(Rys.~\ref{img:gruper2}) wyświetlane po kliknięciu przycisku ,,Dodaj pobyt''. Ustalono następujące dane pobytu: oddział, kod świadczenia. Data przyjęcia i~wypisu przepisuje się automatycznie.
Kolejnym krokiem jest dodanie zdiagnozowanych rozpoznań(Rys.~\ref{img:gruper3}). Aby dodać rozpoznanie do pobytu należy kliknąć przycisk ,,Select''.
Następnie dodano wykonane procedury medyczne. W tym celu należy kliknąć przycisk ,,Dodaj procedurę''(Rys~\ref{img:gruper5}).
Po wybraniu procedury należy ustalić w~jakim terminie została ona wykonana oraz ile razy(Rys.~\ref{img:gruper6}).
Zatwierdzenie przyciskiem ,,OK'' przenosi użytkownika do okienka z~zdefiniowanym pobytem(Rys.~\ref{img:gruper7}).
Kiedy wszystkie dane dla pobytu zostały uzupełnione, zapis pobytu następuje po zatwierdzeniu przyciskiem OK(Rys.~\ref{img:gruper7}).
Widok ,,Gruper'' z~wypełnionymi wszystkimi danymi wejściowymi prawidłowo pozwala użytkownikowi na uruchomienie algorytmu. Należy nacisnąć przycisk ,,Grupuj''(Rys.~\ref{img:gruper8}).

\begin{figure}%[!ht]
\centering
\includegraphics[scale=0.4]{images/gruper1}
\caption[Widok grupera]{Widok grupera - dane wejściowe.}
\label{img:gruper1}
\end{figure}

\begin{figure}%[!ht]
\centering
\includegraphics[scale=0.4]{images/gruper2}
\caption[Widok grupera]{Okno dialogowe - dodawanie pobytu.}
\label{img:gruper2}
\end{figure}

\begin{figure}%[!ht]
\centering
\includegraphics[scale=0.4]{images/gruper3}
\caption[Widok grupera]{Okno dialogowe - wybór rozpoznania.}
\label{img:gruper3}
\end{figure}

\begin{figure}%[!ht]
\centering
\includegraphics[scale=0.4]{images/gruper4}
\caption[Widok grupera]{Okno dialogowe - dodawanie pobytu - rozpoznanie.}
\label{img:gruper4}
\end{figure}

\begin{figure}%[!ht]
\centering
\includegraphics[scale=0.4]{images/gruper5}
\caption[Widok grupera]{Okno dialogowe - wybór procedury.}
\label{img:gruper5}
\end{figure}

\begin{figure}%[!ht]
\centering
\includegraphics[scale=0.4]{images/gruper6}
\caption[Widok grupera]{Okno dialogowe - wybór procedury - data wykonania, ilość.}
\label{img:gruper6}
\end{figure}

\begin{figure}%[!ht]
\centering
\includegraphics[scale=0.4]{images/gruper7}
\caption[Widok grupera]{Okno dialogowe - dodawanie pobytu - uzupełnione procedury i~rozpoznania.}
\label{img:gruper7}
\end{figure}

\begin{figure}%[!ht]
\centering
\includegraphics[scale=0.4]{images/gruper8}
\caption[Widok grupera]{Okno dialogowe - dodawanie pobytu - rozpoznanie.}
\label{img:gruper8}
\end{figure}

\begin{figure}%[!ht]
\centering
\includegraphics[scale=0.4]{images/gruper9}
\caption[Widok grupera]{Wyniki grupowania - Lista wszystkich zaakceptowanych grup.}
\label{img:gruper9}
\end{figure}

Uruchamiany jest algorytm wnioskowania, który wybiera warunki do sprawdzenia, testuje je, a~wyniki zapisuje na listach: Zaakceptowane, Potencjalnie droższe, Potencjalnie tańsze. W tym miejscu można zakończyć przebieg grupowania. Została wybrana optymalna grupa JGP: ,,A49 - Udar mózgu - leczenie > 3 dni'' oraz ustalona została wartość kosztów leczenia 77pkt(Rys.~\ref{img:gruper9}).

Następny krok w~prezentacji działania aplikacji to optymalizacja kosztów leczenia. Przedstawione zostanie narzędzie pozwalające przewidywać wyliczane według NFZ koszty leczenia pacjenta. Przeprowadzony zostanie proces maksymalizacji kosztów leczenia. Na tym etapie należy spojrzeć na zakładkę ,,Potencjalnie droższe''(Rys.~\ref{img:gruper10}). Znajdują się tu grupy JGP, wśród których poszukiwane jest najlepsze rozwiązanie.

\begin{figure}%[!ht]
\centering
\includegraphics[scale=0.4]{images/gruper10}
\caption[Widok grupera]{Wyniki grupowania - Potencjalne (droższe).}
\label{img:gruper10}
\end{figure}

Zauważono, że istnieje grupa JGP dająca możliwość wygenerwoania kosztów w~wysokości 162pkt. Jest to ponad dwukrotnie większa wartość od wyliczonej aktualnie. Należy kliknąć przycisk ,,Szczegóły'' i~sprawdzić powody niezaakceptowania grupy A48(Rys.~\ref{img:gruper11}).

\begin{figure}%[!ht]
\centering
\includegraphics[scale=0.4]{images/gruper11}
\caption[Widok grupera]{Szczegóły niezaakceptowania grupy JGP.}
\label{img:gruper11}
\end{figure}

Z rysunku \ref{img:gruper11} można odczytać następujące powody niezaakceptowania pacjenta:
\begin{itemize}
 \item Wymagane jest zdefiniowanie 2 procedur medycznych,
 \item Wymagany jest czas hospitalizacji powyżej 7 dni.
\end{itemize}

W tym momencie zakres działania aplikacji się kończy. Jest to system doradczy, a decyzję optymalizacyjną podejmuje lekarz. Dla przedstawionego przypadku leczenia lekarz stwierdza, że przedłużenie pobytu na oddziale pacjenta o 1 dzień i~wykonanie dodatkowego badania podwoi koszty leczenia. W tym przypadku lekarz może podjąć decyzję o przedłużeniu leczenia o 1 dzień(Rys.~\ref{img:gruper12}).

\begin{figure}%[!ht]
\centering
\includegraphics[scale=0.4]{images/gruper12}
\caption[Widok grupera]{Widok grupera - poprawione dane wejściowe.}
\label{img:gruper12}
\end{figure}

Dodano kolejną procedurę medyczną, badanie radiologiczne: ,,Arteriografia tętnicy szyjnej wewnętrznej''(Rys.~\ref{img:gruper13}, Rys.~\ref{img:gruper14}).

\begin{figure}%[!ht]
\centering
\includegraphics[scale=0.4]{images/gruper13}
\caption[Widok grupera]{Okno dialogowe - dodawanie procedury.}
\label{img:gruper13}
\end{figure}

\begin{figure}[!ht]
\centering
\includegraphics[scale=0.4]{images/gruper14}
\caption[Widok grupera]{Okno dialogowe - dodawanie pobytu - dodatkowo procedura.}
\label{img:gruper14}
\end{figure}

W wyniku ponownego uruchomienia algorytmu grupowania, otrzymano jako wynik spodziewaną grupę JGP(Rys.~\ref{img:gruper15}).

\begin{figure}%[!ht]
\centering
\includegraphics[scale=0.4]{images/gruper15}
\caption[Widok grupera]{Wyniki grupowania - Lista wszystkich zaakceptowanych grup.}
\label{img:gruper15}
\end{figure}

Wybrana zostaje grupa ,,A48 - Kompleksowe leczenie udarów mózgu > 7 dni w~oddziale udarowym''. Zoptymalizowany koszt leczenia to 162 jednostki punktowe.

%---------------------------------------------------------------------------

\section{Testy integracyjne}
\label{sec:testyIntegracyjne}

Dobrą praktyką zwiększającą współczynnik jakości oprogramowania jest wysokie pokrycie kodu w~testach. Zdecydowano się zatem napisać testy integracyjne klasy JGPService. Stworzono klasę ,,JGPServiceTest'', której zadaniem jest przetestowanie wszystkich głównych ścieżek algorytmu grupowania, gdzie testowana grupa wynikowa będzie pochodzić z~konkretnego warunku kierunkowego. Zatem poszczególne testy są napisane w~ten sposób, aby przetestować każdy z~możliwych 26 warunków kierunkowcyh A-Z.
Przykład przedstawiony w~sekcji \ref{sec:praktyczneUzycieSystemu} został również zaimplementwoany w~testach. Test ten sprawdza poprawne działanie systemu dla warunku kierunkowego E - ,,grupa zdefiniowana procedurą i~rozpoznaniem zasadniczym; może mieć dodatkowe warunki (czas hospitalizacji, wiek)''.
Dla testów integracyjnych przygotowana została testowa baza wiedzy. Przebieg testu jest następujący: ustalane są dane epizodu, uruchamiany jest mechanizm wnioskowania. Następnie sprawdzane są wyniki. Testowana jest poprawność wyników z~listy zaakceptowanych kodów oraz z~listy niezaakceptowanych kodów, wraz z~testami zgodności powodów niezaakceptowania z~wartościami oczekiwanymi.
Kod przykładowego testu integracyjnego zawiera listing \ref{java_test}.

\newpage
\begin{lstlisting}[language=Java,caption={Test integracyjny sprawdzający warunek kierunkowy ,,E''.},label=java_test]
@Test
public void testGrouperE() {
    Episode episode = createTestEpisode(new String[]{"G08"}, new String[]{"88.714"}, 7, 75);
    //run grouper
    JGPGroupResult result = jgpService.group(episode);
    //test accepted
    Assert.assertEquals(1, result.accepted().size());
    JGPResult acceptedJGP = result.accepted().get(0);
    Assert.assertEquals(77.0, acceptedJGP.getValue(), 0.0);
    Assert.assertEquals("A49", acceptedJGP.getJgp().getCode());
    //test NOT accepted
    Assert.assertEquals(1, result.notAccepted().size());
    JGPResult notAcceptedJGP = result.notAccepted().get(0);
    Assert.assertEquals(162.0, notAcceptedJGP.getValue(), 0.0);
    Assert.assertEquals("A48", notAcceptedJGP.getJgp().getCode());
    HospitalReason hospReason = notAcceptedJGP.reasons(HospitalReason.class).get(0);
    Assert.assertEquals(7, hospReason.required().getOver().intValue());
    Assert.assertEquals(TimeUnit.DAY, hospReason.required().getTimeUnit());
}
\end{lstlisting}


\chapter{Podsumowanie i wnioski}
\label{cha:podsumowanie}

%---------------------------------------------------------------------------

\section{Podsumowanie}
\label{sec:podsumowanie}

%---------------------------------------------------------------------------

\section{Możliwość rozwoju}
\label{sec:mozliwoscRozwoju}

%---------------------------------------------------------------------------

\section{Porównanie z innymi istniejącymi rozwiązaniami}
\label{sec:porownanieZinnymiRozwiazaniami}

%---------------------------------------------------------------------------

\section{Wykorzystane biblioteki}
\label{sec:wykorzystaneBiblioteki}

% \chapter{Zakończenie}
\label{cha:zakonczenie}

Metryka oprogramowania – miara pewnej własności oprogramowania lub jego specyfikacji. Termin ten nie ma precyzyjnej definicji i może oznaczać właściwie dowolną wartość liczbową charakteryzującą oprogramowanie
Zestawienie metryk projektu:

191 klas JAVY

7009 LOC

MOOD metrics:
hermetyzacja
AHF - Attribute Hiding Factor - 78,26\%
MHF Method Hiding Factor 23,55\%

dziedziczenie
AIF - Attribute Inheritance Factor71,49\%
MIF Method Inheritance Factor  32,29\%

przekazywanie komunikatów 
CF Coupling Factor 5,47\%

polimorfizm
PF 57,21\%




% itd.
% \appendix
% \include{dodatekA}
% \include{dodatekB}
% itd.

\bibliographystyle{alpha}
\bibliography{bibliografia}
%\begin{thebibliography}{1}
%
%\bibitem{Dil00}
%A.~Diller.
%\newblock {\em LaTeX wiersz po wierszu}.
%\newblock Wydawnictwo Helion, Gliwice, 2000.
%
%\bibitem{Lam92}
%L.~Lamport.
%\newblock {\em LaTeX system przygotowywania dokumentów}.
%\newblock Wydawnictwo Ariel, Krakow, 1992.
%
%\bibitem{Alvis2011}
%M.~Szpyrka.
%\newblock {\em {On Line Alvis Manual}}.
%\newblock AGH University of Science and Technology, 2011.cccccc
%\newblock \\\texttt{http://fm.ia.agh.edu.pl/alvis:manual}.
%
%\end{thebibliography}

\end{document}
