\chapter{Podsumowanie i wnioski}
\label{cha:podsumowanie}

Celem niniejszej pracy było stworzenie aplikacji gruper zgodnie ze specyfikacją NFZ. Aplikacja miała zostać zaprojektowana tak, aby spełniała założenia systemu ekspertowego. Następnie miała zostać rozszerzona tak, aby wspomóc proces optymalizacji kosztów leczenia pacjetna poprzez mechanizm podpowiadający lekarzowi rozwiązania lepsze dla stawianego problemu. Zadanie jest trudne, gdyż wymaga zdobycia i zrozumienia wiedzy z zakresu medycznegeo. Interpretacja dokumentacji, tak zwanych wskazówek do napisania aplikacji typu gruper było zadaniem nietrywialnym. Dostosowanie systemu tak, aby był on systemem ekspertowym wymagało również odpowiedniego podejścia, ponieważ system powinien dalej w 100\% zachowywać się tak jak zdefiniował to NFZ.

System, który stworzyłem jest zgodny z wszystkimi stawianymi na początku założeniami projektowymi. Jest zalążkiem systemu ekspertowego, ponieważ nie wszystkie reguły może zdefiniwoać inżynier. W sekcji~\ref{sec:mozliwoscRozwoju} opisana zostałą wizja możliwosci rozwoju systemu.

%---------------------------------------------------------------------------

\section{Możliwość rozwoju}
\label{sec:mozliwoscRozwoju}

Ze wględu na fakt, że aplikacja została napisana w modelu architektury 3-warstowej, bardzo dobrym rozwinięciem istniejącego rozwiązania byłoby: uruchomienie części odpowiadającej za logikę biznesową i za dostęp do danych na serwerze aplikacyjnym. Warstwę prezentacji możnaby zaimplementować jako webową. Tak, żeby stworzyć system on-line z gruperem.

Duże możliwości rozwoju projektu sprawia fakt, że jest to jedyne rozwiązanie typu openSource na rynku. Tego typu otwarte projekty przyciągają zainteresowanie jednostek dla których jest dedykowane rozwiązanie. Programiści, projektanci aplikacji z branży medycznej mieliby duży wpływ na dalszy rozwój aplikacji. Specjaliści mogliby zgłaszać błędy, tworząc bug-report, oraz proponować własne funkcjonalności, zgłaszając je w systemie z zadaniami. Mogliby oni używać części serwerowej systemu w jednym z modułów swoich złożonych aplikacji do obsługi szpitala. Wtedy nie spada odpowiedzialność na każdą z osobna aplikację w Polsce do obsługi przychodni bądź szpitala tworzenie własnej implementacji grupera. Warto zaznaczyć w ty miejscu, że jeśli lekarz nie posiada dedykowanej aplikacji grupera to jego obowiązkiem jest manualne wyznaczanie grup JGP na podstawie arkusza Excel publikowanego przez NFZ. 
Zakładając, że istniałby w Polsce taki jeden pewny moduł do obsługi rozliczeń kosztów leczenia pacjenta, mógłby on być używany przez wszystkie istniejące aplikacje. Rozwiązania komercyjne musiałyby jedynie podpiąć się do modułu przez zewnętrzne serwisy. Projektanci, architekci i developerzy mieliby dostęp do pełnej dokumentacji tego modułu i mogliby go w pełni wykorzystywać, nie tracąc cennego czasu na implmentację od zera.

Ważnym kierunkiem rozwoju aplikacji jest rozwój w kierunku systemu ekspertowego. Standaryzacja wiedzy, łatwość modyfikowania faktów i reguł systemu sprawiają, że jest on potężnym narzędziem w rękach ekspertów. Częste wprowadzanie zmian w regułach grupowania sprawia, że w standardowym podejściu - programiści muszą wprowadzać zmiany do kodu algorytmu. W podejściu ekspertowym zmiany te mogliby wprowadzać bezpośrednio inżynierowie wiedzy, bez potrzeby ingerencji w kod systemu. Pracownicy NFZ produkujący dokumentację w formie MS-Word(algorytm reguły) i w MS-Excel bazę wiedzy, mogliby po zapoznaniu się z systemem usprawniać go. Wprowadzaliby oni dane do systemu jako inżynierowie wiedzy. Dodawaliby nowe fakty takie jak procedury, rozpoznania, kody jgp. Edytowaliby reguły, zmieniając warunki akceptacji grupy JGP, modyfikując obiekty klasy JGPParameters oraz powiązane z nimi obiekty klasy Agelimit - HospitalLimit. Wszystkie warunki dla reguł mogliby edytować z poziomu dostępu ,,experta'' do systemu.
W takim podejściu wyróżniłbym na początku 2 role: EXPERT i USER. EXPERT miałby dostęp do edycji faktow i reguł. USER do uruchomienia wnioskowania.

Nasuwają się też pomysły na ulepszenie jądra systemu ekspertowego, którym jest mechanizm wnioskowania. Aktualne rozwiązanie zapisuje warunki kierunkowe w bazie jako literki alfabetu. Krokiem w kierunku zapisywania reguł w systemie byłoby zapisywanie skryptów języka Groovy w bazie wiedzy, oraz udostępnienie ich do edycji dla użytkowników z rolą eskperta. Każdy z tych skryptów byłby pobierany i odpalany przez silnik w odpowiednim momencie.
Dalszym rozwinięciem byłoby umieszczenie kolejnych skryptów grooviego w bazie wiedzy, które byłyby odpalane przez silnik wnioskowania. Na przykład wyliczanie wartości dla mechanizmu osobodni. Skryptem zapisanym w języku groovy mogłaby też być część kodu wybierająca odpowiednią ścieżkę poszukiwania(ICD9 lub ICD10). Dzięki takiemu podejściu mielibyśmy uniwersalny system regułowy, którego algorytm mógłby być edytowany bez ingerencji w kod javy. Inżynierowie wiedzy, jeśli nie pracownicy NFZ, to lekarze zainteresowani systemem wprowadzaliby poprawki w kluczowych dla algorytmu wycinkach kodu zapisanych w GROOVY'm, ktory jest językiem bardzo czytelnym, przejrzystym, łatwym do nauczenia i zrozumienia dla każdego. W ten sposób otrzymalibyśmy pełen system ekspertowy usprawniający wszystkie aplikacje medyczne w Polsce z bazą wiedzy kontrolowaną przez NFZ, ktory wprowadzałby poprawki do algorytmu na bieżąco. Każda zewnętrzna aplikacja mogłaby podpiąc się do dedykowanych serwerów i używać grupowania zgodnego z zasadami NFZ.

Kolejnym pomysłem na rozwinięcie aplikacji jest używanie programowania aspektowego. Dla metod sprawdzających warunki, zrealizować zapisywanie powodów niezaakceptowania automatycznie przez aspekty, nie ręcznie jak dotychczas. Dla każdego sprawdzanego warunku byłaby zapisywana pełna lista powodów zaakceptowania lub niezaakceptowania. Takie rozwiązanie dałoby pełny obraz scieżki wyszukiwania dla wszystkich sprawdzanych rozwiązań.

Wydanych zostało ponad 10 wersji algorymtu grupera, ale przedmiotem tej pracy nie było wyszukiwanie drobnych różnic między nimi. Możnaby rozwinąć system wprowadzając wersjonowanie algorytmow i tworząc wszystkie instancje algorytmu, które uruchamiane byłyby w zależności od czasu leczenia, kiedy algorytm był aktualny.


%---------------------------------------------------------------------------

\section{Porównanie z innymi istniejącymi rozwiązaniami}
\label{sec:porownanieZinnymiRozwiazaniami}

Przeanalizowałem konkurencyjne rozwiązania na rynku.

MP-System - potezny system do kompleksowej oblsugi szpitali - nie udalo mi sie zainstalowac wersji DEMO - problemy z systemem windows+ jakas stara wersja ms-sql.

Internetowy OPTYALIZATOR JGP -http://jgp.uhc.com.pl/jgpgrouper/ - CompuGroup Medical Polska - Lublin
rozwiazanie z puntu iwdzenia uzytkownika koncowego - podobne do mojego - 5 darmowych grupowan - platny. Firma posiada osobnych ekspeertow do aktualizacji bazy wiedzy. przeporwadzali 4mln grupowan dla klikuset szpitali. jest to rozwiazanie - bardzo bardzo dobre jakosciowo - ale platne.
w tym systemi jest rozwieniez optymalizacja: z ktorej korzysta 58\% uzytkownikow a 42\% korzysta z samego grupowania. czyli zapotrzebowanie na optymalizator istnieje. moje rozwiazanie jest typu openSource - i sklania sie ku profesjonalnemu systemowi eksperckiemu - anie zwyklej aplikacji typu serwer-klient.


http://smartgruper.pl - plante rozwiazanie - gruper platny i nie dostepna wersja demo. zopisu wynika ze jest to wysokiej jakosci oprogramowanie ktore udstepnia wiele funkcjonalnosci. dostepne 3 wersje:
Wersja On-line. Dostęp indywidualny oraz dla małych placówek medycznych. Dostęp do systemu Smart Gruper on-line.
Wersja Professional. Rozwiązanie dla szpitala bez integracji. Instalacja na serwerze w szpitalu.
Wersja Enterprise. Rozwiązanie dla szpitala z integracją. Instalacja na serwerze w szpitalu oraz integracja z istniejącym systemem informatycznym szpitala. Apliakcja udostpenia rowniez optymalizacje - zapisywanie grupwoan itp funkcjonalnosci.

Zadna z aplikacji nie jest rozwiazaniem typu openSource. Integracja z nfz odbywa sie tylko przez ekspertow przepisujacych dane z komunikatow nfz. 

Przyszłością jest powstajacy system Euro-DRG 7, który docelowo stanowiłby jednolity system JGP dla całej Unii Europejskiej.
http://www.eurodrg.eu/

Nfz stal sie prtnerem systemu eurodrg- i pomgaa przy jego tworzeniu. Byc moze stworzenie platformu systemu ekspertowego dla systemu euro-drg bylaby dobrym kierunkiem rozwoju tego typu rozwiazania, takie rozwiazanie typu opensource- mialoby swoich developelrow nie tylko w poslce. niestety nie okreslona zostala jeszcze specyfikacja systemu. Problemem jest jak zwykle fakt ze kazda firma chce miec monopol na caly kraj lub caly kontynent. co jest przeszkoda w tworzeniu najlepszej jakosci orpogramowania, powstaja systemy skomplikowane, ktorych sie uzywa poniewaz nie ma innej alternatywy. Podobnie jak linux jest ciagle ,,konkurencja''dla winodws tak samo oprogramowanie openSource bedzie uczestniczyc w nieustajacym wsycigu z oprogramowaniem komercyjnym. 

%---------------------------------------------------------------------------

\section{Wykorzystane biblioteki}
\label{sec:wykorzystaneBiblioteki}
Oto listing narzędzi wyjkorzystanych do stworzenia systemu open-grupera:
\begin{table}[h]
 \caption{Wykorzystane narzędzia}
 \small\tt
 \centering
 \vspace{0in}
 \begin{tabular}{|l|p{7cm}|}
 \hline
 GIT & repozytorium kod źródłowy (https://github.com/urbanq/mgr) \\
 \hline
 commons-collections-3.2.1 & Pomocnicze metody dla operacji na kolekcjach \\
 \hline
 aspectj-1.6.10 & Aspekty dla JAVY \\
 \hline
 h2database-1.3.166 & Silnik bazodanowy H2 \\
 \hline
 junit-4.10 & Środowisko testowe \\
 \hline
 jgoodies-2.2.0 & GUI narzędzia pomocnicze \\
 \hline
 jodatime-2.1 & Najlepsza biblioteka do wykonywania obliczeń datowo/czasowych \\
 \hline
 spring-framework-3.0.5 & Spring framework z komponentami Core, JDBC, Security, Beans, Aspects, ORM, Test, TX \\
 \hline
 valkyriercp-1.0 & Biblioteka pozwalająca na wysokopozimowe programowanie GUI \\
 \hline
 \end{tabular}
\end{table}

