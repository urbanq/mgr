\chapter{Podsumowanie i wnioski}
\label{cha:podsumowanie}

Celem niniejszej pracy było stworzenie aplikacji ,,Gruper'' zgodnie ze specyfikacją NFZ. Aplikacja została zaprojektowana w ten sposób, aby spełniała założenia systemu ekspertowego. Następnie została rozszerzona tak, aby wspomóc proces optymalizacji kosztów leczenia pacjetna. Zadanie wymagało zdobycia i zrozumienia wiedzy z zakresu medycznegeo. Dostosowanie systemu tak, aby spełniał założenia systemu ekspertowego wymagało również odpowiedniego podejścia, ponieważ powinien on działać nie naruszając specyfikacji zdefiniowanej przez NFZ.

System, który stworzono jest zgodny z wszystkimi stawianymi na początku założeniami projektowymi. Jest on zalążkiem systemu ekspertowego, ponieważ inżynier może definiować reguły tylko w wydzielonym obszarze. W sekcji~\ref{sec:mozliwoscRozwoju} opisana została wizja możliwosci rozwoju systemu.

%---------------------------------------------------------------------------

\section{Możliwość rozwoju}
\label{sec:mozliwoscRozwoju}

Ze wględu na fakt, że aplikacja została napisana w modelu architektury warstowej, bardzo dobrym rozwinięciem istniejącego rozwiązania byłoby uruchomienie części odpowiadającej za logikę biznesową i za dostęp do danych na serwerze aplikacyjnym. Warstwę prezentacji można zaimplementować jako aplikację webową. Rozszerzenie to pozwala stworzyć system on-line z algorytmem ,,Grupera''.

Duże możliwości rozwoju projektu sprawia fakt, że jest to jedyne rozwiązanie typu ,,open source'' na rynku. Tego typu otwarte projekty przyciągają zainteresowanie jednostek, dla których rozwiązanie to jest dedykowane. Uruchomienie systemu do zgłaszania błędów i tworzenia własnych propozycji nowych funkcjonalności jest dobrym krokiem w kierunku rozwoju oprogramowania typu ,,open source''. Programiści, projektanci oraz architekci systemów z branży medycznej mogliby mieć swój wpływ na dalszy rozwój aplikacji.

Aplikacje zewnętrzne mogłyby korzystać z serwerowej części systemu. Takie rozwiązanie wspomaga proces tworzenia oprogramowania medycznego, ściągając z aplikacji do osługi szpitala odpowiedzialność za tworzenie własnej implementacji algorytmu ,,Grupera''. Warto zaznaczyć w tym miejscu, że jeśli lekarz nie posiada dedykowanej aplikacji ,,Grupera'' to jego obowiązkiem jest wyznaczanie grup JGP na podstawie arkusza MS-Excel opublikowanego przez NFZ\cite{plik_parametryzujacy}. 
Zakładając, że istniałby ogólnodostępny moduł do obsługi rozliczeń kosztów leczenia pacjenta, mógłby on być używany przez większość istniejących aplikacji. Rozwiązania komercyjne musiałyby jedynie połączyć się z modułem używając zewnętrznych serwisów. Projektanci, architekci i programiści mieliby dostęp do pełnej dokumentacji tego modułu i mogliby go w pełni wykorzystywać, nie tracąc cennego czasu na implmentację algorytmu od zera.

Ważnym kierunkiem rozwoju aplikacji byłby rozwój w kierunku systemu ekspertowego. Standaryzacja wiedzy, łatwość modyfikowania faktów i reguł systemu sprawiają, że jest on potężnym narzędziem w rękach ekspertów. Częste wprowadzanie zmian w regułach grupowania sprawia, że w standardowym podejściu, programiści muszą wprowadzać zmiany do kodu algorytmu. W podejściu ekspertowym zmiany te mogliby wprowadzać bezpośrednio inżynierowie wiedzy, bez potrzeby ingerencji w kod systemu.
%Pracownicy NFZ produkujący dokumentację w formie MS-Word(algorytm i reguły) i w MS-Excel bazę wiedzy, mogliby po zapoznaniu się z systemem używać go do tworzenia reguł w dedykowanym do tego języku.
%Wprowadzaliby oni dane do systemu jako inżynierowie wiedzy. Dodawaliby nowe fakty takie jak procedury, rozpoznania, kody jgp. Edytowaliby reguły, zmieniając warunki akceptacji grupy JGP, modyfikując obiekty klasy JGPParameters oraz powiązane z nimi obiekty klasy np. Agelimit, HospitalLimit. 
%Wszystkie warunki dla reguł mogliby edytować z poziomu dostępu ,,experta'' do systemu.
W takim podejściu zaleca się wyróżnienie dwóch ról: eksperta i użytkownia. Ekspert miałby dostęp do edycji faktów i reguł, a użytkownik do uruchomienia algorytmu wnioskowania.

Następną ścieżką rozwoju oprogramowania jest modernizacja jądra systemu ekspertowego, którym jest mechanizm wnioskowania. Aktualne rozwiązanie zapisuje warunki kierunkowe w bazie jako literki alfabetu. Krokiem w kierunku zapisywania pełnych reguł w systemie byłoby zapisywanie skryptów języka Groovy w bazie wiedzy, oraz udostępnienie ich do edycji dla użytkowników z rolą eskperta. Każdy z tych skryptów byłby pobierany i uruchomiany przez silnik w odpowiednim momencie.
%Dalszym rozwinięciem byłoby umieszczenie kolejnych skryptów Groovy w bazie wiedzy, które byłyby również uruchamiane przez silnik wnioskowania. Skryptem zapisanym w języku Groovy mogłaby być część kodu wybierająca odpowiednią ścieżkę poszukiwania(ICD9 lub ICD10) lub kod wyliczający wartości dla mechanizmu osobodni. 
Takie podejście tworzy uniwersalny system regułowy, którego algorytm mógłby być edytowany bez ingerencji w kod JAVY.
%Inżynierowie wiedzy, jeśli nie pracownicy NFZ, to lekarze zainteresowani systemem wprowadzaliby poprawki w kluczowych dla algorytmu wycinkach kodu zapisanych w Groovy'm, ktory jest językiem bardzo czytelnym, przejrzystym, łatwym do nauczenia i zrozumienia dla każdego. W ten sposób otrzymalibyśmy spójny system ekspertowy usprawniający wszystkie aplikacje medyczne w Polsce z bazą wiedzy kontrolowaną przez NFZ, którego zadaniem byłoby wprowadzanie na bieżąco poprawek do algorytmu. Każda zewnętrzna aplikacja mogłaby podpiąc się do dedykowanych serwerów i używać grupowania zgodnego z zasadami NFZ.

Kolejnym pomysłem na rozwinięcie aplikacji jest używanie programowania aspektowego do zapisywania powodów niezaakceptowania warunku.
%Dla metod sprawdzających warunki można zrealizować zapisywanie powodów niezaakceptowania automatycznie poprzez aspekty, nie ręcznie jak dotychczas. 
Dla każdego sprawdzanego warunku byłaby zapisywana pełna lista powodów zaakceptowania lub niezaakceptowania.
Takie rozwiązanie dałoby całościowy obraz scieżki wyszukiwania dla wszystkich sprawdzanych rozwiązań.

Wydanych zostało ponad 10 wersji algorymtu grupera, ale przedmiotem tej pracy nie było wyszukiwanie drobnych różnic między nimi. Można rozwinąć system wprowadzając wersjonowanie algorytmów i tworząc wszystkie instancje algorytmu, które uruchamiane byłyby w zależności od daty hospitalizacji, dla której algorytm jest aktualny.

%---------------------------------------------------------------------------

\section{Porównanie z innymi istniejącymi rozwiązaniami}
\label{sec:porownanieZinnymiRozwiazaniami}

Przeanalizowano konkurencyjne rozwiązania na rynku. Najpopularniejsze aplikacje, postanowiono zainstalować, uruchomić i porównać z rozwiązaniem stworzonym na potrzeby pracy. %A oto listing systemów medycznych zawierających algorytm grupera:

\textbf{MP-System} - złożony system do kompleksowej obsługi szpitali. Już w trakcie instalacji wersji demo pojawiły się błędy dotyczące starej wersji serwera MS-SQL(dostarczanej wraz z aplikacją). Niestety nawet po kilku-godzinnych trudach nie udało się zainstalować i uruchomić aplikacji. Skomplikowany proces instalacji uniemożliwił zainstalowanie systemu Medycyny Praktycznej.

\textbf{Internetowy OPTYMALIZATOR JGP} CompuGroup Medical Polska Lublin - wersja demo dostępna pod adresem http://jgp.uhc.com.pl/jgpgrouper(dostęp 12.06.2012).
Aplikacja umożliwia przeporawdzenie pięciu darmowych grupowań dziennie. Powyżej tej liczby jest aplikacja jest płatna. Według statystyk zamieszonych na stronie przy użyciu aplikacji przeprowadzono 4mln grupowań dla kilkuset szpitali.%Jest to rozwiązanie wysokiej jakości, ale płatne.
System ten posiada również moduł do optymalizacji, z którego korzysta 58\% użytkowników(pozostałe 42\% korzysta z samego algorytmu grupowania).

\textbf{SMARTGRUPER} - http://smartgruper.pl(dostęp 12.06.2012). Jest to komercyjne rozwiązanie, nie posiadające wersji demo. Z opisu na stronie projektu wynika, że jest to oprogramowanie, które posiada wiele funkcjonalności poza standardowym ,,Gruperem'', na przykład moduł optymalizacji JGP, zapisywanie grupowań. Dostępne są 3 wersje produktu:
\begin{itemize}
 \item Wersja On-line. Dostęp indywidualny oraz dla małych placówek medycznych. Dostęp do systemu Smart Gruper on-line.
 \item Wersja Professional. Rozwiązanie dla szpitala bez integracji. Instalacja na serwerze w szpitalu.
 \item Wersja Enterprise. Rozwiązanie dla szpitala z integracją. Instalacja na serwerze w szpitalu oraz integracja z istniejącym systemem informatycznym szpitala.
\end{itemize}

Każdy z powyżej wymienionych systemów jest rozwiązaniem komercyjnym. System stworzony w ramach tej pracy jest aktualnie jedynym rozwiązaniem typu ,,open source'' na rynku oraz jedynym, które wnosi możliwości jakie daje profesjonalny system ekspertowy.

Przyszłością systemów do rozliczania kosztów leczenia pacjenta jest powstający system \mbox{Euro-DRG~7}, który docelowo stanowiłby jednolity system JGP dla całej Unii Europejskiej(http://www.eurodrg.eu).
NFZ stał się partnerem systemu Euro-DRG i aktywnie uczestniczy w jego tworzeniu. Być może stworzenie platformy systemu ekspertowego dla systemu Euro-DRG byłoby dobrym kierunkiem rozwoju tego typu rozwiązania. Jako projekt ,,open source'' posiadałby on swoich aktywnych uczestników nie tylko w Polsce, ale w całej Unii Europejskiej. Niestety aktualnie nie określona została jeszcze specyfikacja systemu. 
\begin{comment}
Problematyczne jest podejście wszystkich firm startujących w przetargach, ponieważ każda z nich chce mieć monopol na cały obszar zastosowania systemu. Jest to w mojej opinii poważna przeszkoda w tworzeniu najlepszej jakości oprogramowania, ponieważ w jej wyniku powstają systemy skomplikowane, ktorych używa się tylko dlatego, że nie ma innej alternatywy. Podobnie jak system operacyjny Linux jest ciągle ,,konkurencją'' dla systemu Winodws, tak samo oprogramowanie openSource w każdej dziedzinie będzie uczestniczyć w nieustającym wyścigu z oprogramowaniem komercyjnym. 
\end{comment}
%---------------------------------------------------------------------------

\section{Wykorzystane biblioteki}
\label{sec:wykorzystaneBiblioteki}
Tabela \ref{tab:tools} zawiera listę narzędzi wykorzystanych do stworzenia systemu.
\begin{table}[h]
 \caption{Wykorzystane narzędzia}
 \small\tt
 \centering
 \vspace{0in}
 \begin{tabular}{|l|p{7cm}|}
 \hline
 GIT & repozytorium kod źródłowy (https://github.com/urbanq/mgr) \\
 \hline
 commons-collections-3.2.1 & Pomocnicze metody dla operacji na kolekcjach \\
 \hline
 aspectj-1.6.10 & Aspekty dla JAVY \\
 \hline
 h2database-1.3.166 & Silnik bazodanowy H2 \\
 \hline
 junit-4.10 & Środowisko testowe \\
 \hline
 jgoodies-2.2.0 & GUI narzędzia pomocnicze \\
 \hline
 jodatime-2.1 & Najlepsza biblioteka do wykonywania obliczeń datowo/czasowych \\
 \hline
 spring-framework-3.0.5 & Spring framework z komponentami Core, JDBC, Security, Beans, Aspects, ORM, Test, TX \\
 \hline
 valkyriercp-1.0 & Biblioteka pozwalająca na wysokopozimowe programowanie GUI \\
 \hline
 \end{tabular}
 \label{tab:tools}
\end{table}

%---------------------------------------------------------------------------

\section{Jakość oprogramowania}
\label{sec:jakoscOprogramowania}

Zestawienie wartości liczbowych charakteryzujące oprogramowanie jest nazywane metryką oprogramowania. Dla projektu wykonanego w ramach pracy obliczyłem następujące wartości metryk chrakteryzujące jego wielkość:
\begin{itemize}
 \item 7009 LOC(Lines of code) - ilość linijek kodu źródłowego
 \item 191 C(Classes) - ilość zdefiniowanych klas JAVY.
\end{itemize}

Następne zestawienie metryk charakteryzujące jakość napisanego kodu to MOOD - Metrics for Object-Oriented Design. Metryki te służą do wyrażenia w procentach cech charakteryzujących dla programowania obiektowego:
\begin{itemize}
 \item 78,26\% AHF - Attribute Hiding Factor - hermetyzacja atrybutów
 \item 23,55\% MHF - Method Hiding Factor - hermetyzacja metod
 \item 71,49\% AIF - Attribute Inheritance Factor - dziedziczenie atrybutów
 \item 32,29\% MIF - Method Inheritance Factor - dziedziczenie metod
 \item 5,47\% CF - Coupling Factor - przekazywanie komunikatów 
 \item 57,21\% PF - polimorfizm
\end{itemize}

