\chapter{Wstęp}
\label{cha:wstep}

\LaTeX~jest systemem składu umożliwiającym tworzenie dowolnego typu dokumentów (w~szczególności naukowych i technicznych) o wysokiej jakości typograficznej (\cite{Dil00}, \cite{Lam92}). Wysoka jakość składu jest niezależna od rozmiaru dokumentu -- zaczynając od krótkich listów do bardzo grubych książek. \LaTeX~automatyzuje wiele prac związanych ze składaniem dokumentów np.: referencje, cytowania, generowanie spisów (treśli, rysunków, symboli itp.) itd.

\LaTeX~jest zestawem instrukcji umożliwiających autorom skład i wydruk ich prac na najwyższym poziomie typograficznym. Do formatowania dokumentu \LaTeX~stosuje \TeX a (wymiawamy 'tech' -- greckie litery $\tau$, $\epsilon$, $\chi$). Korzystając z~systemu składu \LaTeX~mamy za zadanie przygotować jedynie tekst źródłowy, cały ciężar składania, formatowania dokumentu przejmuje na siebie system.

%---------------------------------------------------------------------------

\section{Cele pracy}
\label{sec:celePracy}

Celem poniższej pracy jest stworzenie aplikacji typu gruper według wytycznych Narodowego Funduszu Zdrowia. Aplikacja będzie systemem ekspertowym opartym na bazie wiedzy dostarczonej przez NFZ. Algorytm grupowania będzie własną implementacją systemu regułowego w języku JAVA, który będzie odpowiadał algorytmowi zapropowanemu przez NFZ. Program będzie jedynym rozwiązaniem typu OpenSource na rynku. Poruszony zostanie również problem optymalizacji procesu grupowania, czyli maksymalizacji kosztów leczenia pacjenta. Oczywistym faktem jest, że każda placówka medyczna chce dostać jak największy zwrot poniesionych środków od płatnika. Na końcu przedstawione zostaną realne przypadki dla konkretnych placówek oraz konkretnych przypadków leczenia dla których użyto aplikacji. Napisane zostaną testy integracyjne, sprawdzające poprawność działania algorytmu grupera dla wszystkich zdefiniowanych reguł. 
Podsumowując:
\begin{enumerate}
\item Implementacja algorytmu grupera NFZ
\item Stworzenie własnego systemu ekspertowego dla rozwiązania
\item Optymalizacja kosztów leczenia, rozszerzenie grupera
\item Jedyne na rynku rozwiązanie OpenSource
\item Testy integracyjne, realne przypadki użycia
\end{enumerate}


%---------------------------------------------------------------------------

\section{Zawartość pracy}
\label{sec:zawartoscPracy}

W rodziale~\ref{cha:wprowadzenie} przedstawiono wszystkie podstawowe informacje oraz zdefiniowano słownik podstawowych pojęć. Wyjaśniona została istota aplikacji grupera oraz powszechnego na całym świecie systemu Jednorodnych Grup Pacjentów. Przedstawiona została również bardzo skrócona wersja historii powstania systemu JGP.

W rodziale~\ref{cha:rozwiazanie} mamy program

W rodziale~\ref{cha:prezentacja} mamy prezentacje programu

Podsumowanie pracy znajduje się w rozdziale ~\ref{cha:podsumowanie} - są tam....

