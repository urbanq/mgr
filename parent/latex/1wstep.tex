\chapter{Wstęp}
\label{cha:wstep}

Rozliczanie kosztów leczenia pacjenta jest problemem skomplikowanym. Z jednej strony dążymy do rozwiązania, w~którym kluczowym czynnikiem jest zdrowie pacjenta. Aspekt medyczny brany jest pod uwagę na pierwszym miejscu. Jednak z drugiej strony równie ważnym jest aspekt finansowy. Nieprawidłowo działający system rozliczeń ma bezpośredni wpływ na aspekt medyczny, co przekłada się na niską jakość świadczonych usług\cite{kozierkiewicz_jgp}. Należy więc oczekiwać, że przyjęta przez płatnika metoda finansowania świadczeń opieki zdrowotnej powinna dążyć do zapewnienia obiektywnego pomiaru kosztów leczenia w~jednostkach szpitalnych. Jednocześnie powinna motywować świadczeniodawców do zwiększania dostępności do odpowiedniej jakości świadczeń, jak również ograniczać nieuzasadniony wzrost kosztów opieki zdrowotnej.

Dotychczasowe doświadczenia światowe wskazują, że najlepszą metodą finansowania świadczeń medycznych jest system Jednorodnych Grup Pacjentów\cite{kozierkiewicz_jgp}. Łączy on zarówno aspekt medyczny jak i finansowy. Istotnym założeniem tego systemu jest jego otwarty charakter. Corocznie każdy z krajów, w~których obowiązuje system JGP przyjmuje nową wersję systemu, który jest ciągle doskonalony i rozwijany.

%---------------------------------------------------------------------------

\section{Cele pracy}
\label{sec:celePracy}

Celem poniższej pracy jest stworzenie aplikacji według wytycznych Narodowego Funduszu Zdrowia\cite{plik_parametryzujacy}\cite{algorytm_grupera}. Aplikacja będzie systemem ekspertowym opartym na bazie wiedzy dostarczonej przez NFZ\cite{plik_parametryzujacy}. Algorytm grupowania będzie własną implementacją systemu regułowego w~języku JAVA, który będzie odpowiadał algorytmowi zapropowanemu przez NFZ\cite{algorytm_grupera}. Program będzie jedynym rozwiązaniem typu ,,open source'' na rynku. Poruszony zostanie również problem optymalizacji procesu grupowania, czyli maksymalizacji kosztów leczenia pacjenta. Należy podkreślić, że każda placówka medyczna chce dostać jak największy zwrot poniesionych środków od płatnika. Na końcu pracy przedstawione zostaną realne przypadki użycia systemu dla konkretnych przypadków leczenia. Napisane zostaną testy integracyjne, sprawdzające poprawność działania algorytmu dla wszystkich zdefiniowanych reguł. 
Podsumowując, stawiane są następujące cele pracy:
\begin{enumerate}
\item Implementacja algorytmu ,,Gruper JGP''.
\item Stworzenie własnej implementacji systemu ekspertowego dla rozwiązania.
\item Optymalizacja kosztów leczenia.
\item Oprogramowanie udostępnione zostanie na licencji ,,open source''.
\item Napisanie testów integracyjnych(realne przypadki użycia).
\end{enumerate}

%---------------------------------------------------------------------------

\section{Zawartość pracy}
\label{sec:zawartoscPracy}

W rodziale~\ref{cha:wprowadzenie} przedstawiono wszystkie podstawowe informacje oraz zdefiniowano słownik podstawowych pojęć. Wyjaśniona została istota aplikacji ,,Gruper'' oraz powszechnego na całym świecie systemu Jednorodnych Grup Pacjentów. Przedstawiona została również historia systemu JGP w bardzo skróconej wersji.

W rodziale~\ref{cha:rozwiazanie} został szczegółowo opisany system, który został stworzony w~ramach tej pracy. W~tym rozdziale znajduje się opis architektury systemu, modelu danych, następnie wyjaśniony został szczegółowo algorytm grupowania. Opis systemu ekspertowego własnej implementacji odpowiadającego algorytmowi grupowania znajduje się w~podrozdziale~\ref{sec:systemEkspertowy}.

Rozdział~\ref{cha:prezentacja} zawiera prezentację działania systemu na konkretnym przykładzie. Przedstawiony został przypadek użycia aplikacji, który zawiera rozliczenie kosztów hospitalizacji oraz przeprowadzenie optymalizacji otrzymanych wyników.

Podsumowanie pracy znajduje się w~rozdziale ~\ref{cha:podsumowanie} - są tam informacje o możliwościach rozwoju aplikacji, porównanie z istniejącymi na rynku rozwiązaniami oraz zestawienie wszystkich użytych narzędzi i bibliotek.

